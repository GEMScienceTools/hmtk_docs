%%%%%%%%%%%%%%%%%%%%%%%%%%%%%%%%%%%%%%%%%
% The Legrand Orange Book
% LaTeX Template
% Version 1.4 (12/4/14)
%
% This template has been downloaded from:
% http://www.LaTeXTemplates.com
%
% Original author:
% Mathias Legrand (legrand.mathias@gmail.com)
%
% License:
% CC BY-NC-SA 3.0 (http://creativecommons.org/licenses/by-nc-sa/3.0/)
%
% Compiling this template:
% This template uses biber for its bibliography and makeindex for its index.
% When you first open the template, compile it from the command line with the 
% commands below to make sure your LaTeX distribution is configured correctly:
%
% 1) pdflatex hmtk_tutorial
% 2) makeindex hmtk_tutorial.idx -s StyleInd.ist
% 3) biber hmtk_tutorial
% 4  makeglossaries hmtk_tutorial
% 4) pdflatex hmtk_tutorial x 2
%
% After this, when you wish to update the bibliography/index use the appropriate
% command above and make sure to compile with pdflatex several times 
% afterwards to propagate your changes to the document.
%
% This template also uses a number of packages which may need to be
% updated to the newest versions for the template to compile. It is strongly
% recommended you update your LaTeX distribution if you have any
% compilation errors.
%
% Important note:
% Chapter heading images should have a 2:1 width:height ratio,
% e.g. 920px width and 460px height.
%
%%%%%%%%%%%%%%%%%%%%%%%%%%%%%%%%%%%%%%%%%

%----------------------------------------------------------------------------------------
%	PACKAGES AND OTHER DOCUMENT CONFIGURATIONS
%----------------------------------------------------------------------------------------

\documentclass[11pt,fleqn]{book} % Default font size and left-justified equations

\usepackage[top=3cm,bottom=3cm,left=3.2cm,right=3.2cm,headsep=10pt,a4paper]{geometry} % Page margins

\usepackage{xcolor} % Required for specifying colors by name
\definecolor{ocre}{RGB}{243,102,25} % Define the orange color used for highlighting throughout the book
\usepackage{setspace}

% Font Settings
\usepackage{avant} % Use the Avantgarde font for headings
%\usepackage{times} % Use the Times font for headings
\usepackage{mathptmx} % Use the Adobe Times Roman as the default text font together with math symbols from the Sym­bol, Chancery and Com­puter Modern fonts

\usepackage{microtype} % Slightly tweak font spacing for aesthetics
\usepackage[utf8]{inputenc} % Required for including letters with accents
\usepackage[T1]{fontenc} % Use 8-bit encoding that has 256 glyphs

% Bibliography
\usepackage{csquotes}
\usepackage{cprotect}
\usepackage[style=alphabetic,
            sorting=nyt,
            sortcites=true,
            natbib=true,
            style=authoryear,
            maxcitenames=2,
            maxbibnames=100,
            autopunct=true,
            babel=hyphen,
            hyperref=true,
            doi=true,
            abbreviate=false,
            backref=true,
            uniquename=false,
            uniquelist=false,
            backend=biber]{biblatex}
\addbibresource{./bibliography/hazard.bib} % BibTeX bibliography file
\defbibheading{bibempty}{}

% Figure caption settings
\usepackage[textfont=it,margin=10pt,font=small,labelfont=bf,labelsep=endash]{caption}
\usepackage{subcaption}
\usepackage{rotating}

% Table
\usepackage{color, colortbl}
\definecolor{almond}{rgb}{0.94, 0.87, 0.8}
\definecolor{ashgrey}{rgb}{0.7, 0.75, 0.71}
\definecolor{anti-flashwhite}{rgb}{0.95, 0.95, 0.96}

% Index
\usepackage{calc} % For simpler calculation - used for spacing the index letter headings correctly
\usepackage{makeidx} % Required to make an index
\makeindex % Tells LaTeX to create the files required for indexing

\usepackage{todonotes}

%
% Package to create a glossary - It must be uploaded after hyperref
% to produce the glossary: makeglossaries OQB
\usepackage[acronym,nonumberlist,style=altlist]{glossaries}
\glstoctrue
\makeglossaries

% package for bold symbols
\usepackage{bm}

%----------------------------------------------------------------------------------------
% Insert the commands.tex file which contains the majority of the structure 
% behind the template
%----------------------------------------------------------------------------------------
%	VARIOUS REQUIRED PACKAGES
%----------------------------------------------------------------------------------------

\usepackage{titlesec} % Allows customization of titles

\usepackage{graphicx} % Required for including pictures
\graphicspath{{Pictures/}} % Specifies the directory where pictures are stored

\usepackage{tikz} % Required for drawing custom shapes

\usepackage[english]{babel} % English language/hyphenation

\usepackage{enumitem} % Customize lists
\setlist{nolistsep} % Reduce spacing between bullet points and numbered lists

\usepackage{booktabs} % Required for nicer horizontal rules in tables

\usepackage{eso-pic} % Required for specifying an image background in the title page
\usepackage{listings} % Required for embedding code snippets
\usepackage{fancyvrb} % Verbatim environment

%----------------------------------------------------------------------------------------
%	MAIN TABLE OF CONTENTS
%----------------------------------------------------------------------------------------

\usepackage{titletoc} % Required for manipulating the table of contents

\contentsmargin{0cm} % Removes the default margin
% Chapter text styling
\titlecontents{chapter}[1.25cm] % Indentation
{\addvspace{15pt}\large\sffamily\bfseries} % Spacing and font options for chapters
{\color{ocre!60}\contentslabel[\Large\thecontentslabel]{1.25cm}\color{ocre}} % Chapter number
{}  
{\color{ocre!60}\normalsize\sffamily\bfseries\;\titlerule*[.5pc]{.}\;\thecontentspage} % Page number
% Section text styling
\titlecontents{section}[1.25cm] % Indentation
{\addvspace{5pt}\sffamily\bfseries} % Spacing and font options for sections
{\contentslabel[\thecontentslabel]{1.25cm}} % Section number
{}
{\sffamily\hfill\color{black}\thecontentspage} % Page number
[]
% Subsection text styling
\titlecontents{subsection}[1.25cm] % Indentation
{\addvspace{1pt}\sffamily\small} % Spacing and font options for subsections
{\contentslabel[\thecontentslabel]{1.25cm}} % Subsection number
{}
{\sffamily\;\titlerule*[.5pc]{.}\;\thecontentspage} % Page number
[] 

%----------------------------------------------------------------------------------------
%	MINI TABLE OF CONTENTS IN CHAPTER HEADS
%----------------------------------------------------------------------------------------

% Section text styling
\titlecontents{lsection}[0em] % Indendating
{\footnotesize\sffamily} % Font settings
{}
{}
{}

% Subsection text styling
\titlecontents{lsubsection}[.5em] % Indentation
{\normalfont\footnotesize\sffamily} % Font settings
{}
{}
{}
 
%----------------------------------------------------------------------------------------
%	PAGE HEADERS
%----------------------------------------------------------------------------------------

\usepackage{fancyhdr} % Required for header and footer configuration

\pagestyle{fancy}
\renewcommand{\chaptermark}[1]{\markboth{\sffamily\normalsize\bfseries\chaptername\ \thechapter.\ #1}{}} % Chapter text font settings
\renewcommand{\sectionmark}[1]{\markright{\sffamily\normalsize\thesection\hspace{5pt}#1}{}} % Section text font settings
\fancyhf{} \fancyhead[LE,RO]{\sffamily\normalsize\thepage} % Font setting for the page number in the header
\fancyhead[LO]{\rightmark} % Print the nearest section name on the left side of odd pages
\fancyhead[RE]{\leftmark} % Print the current chapter name on the right side of even pages
\renewcommand{\headrulewidth}{0.5pt} % Width of the rule under the header
\addtolength{\headheight}{2.5pt} % Increase the spacing around the header slightly
\renewcommand{\footrulewidth}{0pt} % Removes the rule in the footer
\fancypagestyle{plain}{\fancyhead{}\renewcommand{\headrulewidth}{0pt}} % Style for when a plain pagestyle is specified

% Removes the header from odd empty pages at the end of chapters
\makeatletter
\renewcommand{\cleardoublepage}{
\clearpage\ifodd\c@page\else
\hbox{}
\vspace*{\fill}
\thispagestyle{empty}
\newpage
\fi}

%----------------------------------------------------------------------------------------
%	THEOREM STYLES
%----------------------------------------------------------------------------------------

\usepackage{amsmath,amsfonts,amssymb,amsthm} % For math equations, theorems, symbols, etc

\newcommand{\intoo}[2]{\mathopen{]}#1\,;#2\mathclose{[}}
\newcommand{\ud}{\mathop{\mathrm{{}d}}\mathopen{}}
\newcommand{\intff}[2]{\mathopen{[}#1\,;#2\mathclose{]}}
\newtheorem{notation}{Notation}[chapter]

%%%%%%%%%%%%%%%%%%%%%%%%%%%%%%%%%%%%%%%%%%%%%%%%%%%%%%%%%%%%%%%%%%%%%%%%%%%
%%%%%%%%%%%%%%%%%%%% dedicated to boxed/framed environements %%%%%%%%%%%%%%
%%%%%%%%%%%%%%%%%%%%%%%%%%%%%%%%%%%%%%%%%%%%%%%%%%%%%%%%%%%%%%%%%%%%%%%%%%%
\newtheoremstyle{ocrenumbox}% % Theorem style name
{0pt}% Space above
{0pt}% Space below
{\normalfont}% % Body font
{}% Indent amount
{\small\bf\sffamily\color{ocre}}% % Theorem head font
{\;}% Punctuation after theorem head
{0.25em}% Space after theorem head
{\small\sffamily\color{ocre}\thmname{#1}\nobreakspace\thmnumber{\@ifnotempty{#1}{}\@upn{#2}}% Theorem text (e.g. Theorem 2.1)
\thmnote{\nobreakspace\the\thm@notefont\sffamily\bfseries\color{black}---\nobreakspace#3.}} % Optional theorem note
\renewcommand{\qedsymbol}{$\blacksquare$}% Optional qed square

\newtheoremstyle{blacknumex}% Theorem style name
{5pt}% Space above
{5pt}% Space below
{\normalfont}% Body font
{} % Indent amount
{\small\bf\sffamily}% Theorem head font
{\;}% Punctuation after theorem head
{0.25em}% Space after theorem head
{\small\sffamily{\tiny\ensuremath{\blacksquare}}\nobreakspace\thmname{#1}\nobreakspace\thmnumber{\@ifnotempty{#1}{}\@upn{#2}}% Theorem text (e.g. Theorem 2.1)
\thmnote{\nobreakspace\the\thm@notefont\sffamily\bfseries---\nobreakspace#3.}}% Optional theorem note

\newtheoremstyle{blacknumbox} % Theorem style name
{0pt}% Space above
{0pt}% Space below
{\normalfont}% Body font
{}% Indent amount
{\small\bf\sffamily}% Theorem head font
{\;}% Punctuation after theorem head
{0.25em}% Space after theorem head
{\small\sffamily\thmname{#1}\nobreakspace\thmnumber{\@ifnotempty{#1}{}\@upn{#2}}% Theorem text (e.g. Theorem 2.1)
\thmnote{\nobreakspace\the\thm@notefont\sffamily\bfseries---\nobreakspace#3.}}% Optional theorem note

%%%%%%%%%%%%%%%%%%%%%%%%%%%%%%%%%%%%%%%%%%%%%%%%%%%%%%%%%%%%%%%%%%%%%%%%%%%
%%%%%%%%%%%%% dedicated to non-boxed/non-framed environements %%%%%%%%%%%%%
%%%%%%%%%%%%%%%%%%%%%%%%%%%%%%%%%%%%%%%%%%%%%%%%%%%%%%%%%%%%%%%%%%%%%%%%%%%
\newtheoremstyle{ocrenum}% % Theorem style name
{5pt}% Space above
{5pt}% Space below
{\normalfont}% % Body font
{}% Indent amount
{\small\bf\sffamily\color{ocre}}% % Theorem head font
{\;}% Punctuation after theorem head
{0.25em}% Space after theorem head
{\small\sffamily\color{ocre}\thmname{#1}\nobreakspace\thmnumber{\@ifnotempty{#1}{}\@upn{#2}}% Theorem text (e.g. Theorem 2.1)
\thmnote{\nobreakspace\the\thm@notefont\sffamily\bfseries\color{black}---\nobreakspace#3.}} % Optional theorem note
\renewcommand{\qedsymbol}{$\blacksquare$}% Optional qed square
\makeatother

% Defines the theorem text style for each type of theorem to one of the three styles above
\newcounter{dummy} 
\numberwithin{dummy}{section}
\theoremstyle{ocrenumbox}
\newtheorem{theoremeT}[dummy]{Theorem}
\newtheorem{problem}{Problem}[chapter]
\newtheorem{exerciseT}{Exercise}[chapter]
\theoremstyle{blacknumex}
\newtheorem{exampleT}{Example}[chapter]
\theoremstyle{blacknumbox}
\newtheorem{vocabulary}{Vocabulary}[chapter]
\newtheorem{definitionT}{Definition}[section]
\newtheorem{corollaryT}[dummy]{Corollary}
\theoremstyle{ocrenum}
\newtheorem{proposition}[dummy]{Proposition}

%----------------------------------------------------------------------------------------
%	DEFINITION OF COLORED BOXES
%----------------------------------------------------------------------------------------

\RequirePackage[framemethod=default]{mdframed} % Required for creating the theorem, definition, exercise and corollary boxes

% Theorem box
\newmdenv[skipabove=7pt,
skipbelow=7pt,
backgroundcolor=black!5,
linecolor=ocre,
innerleftmargin=5pt,
innerrightmargin=5pt,
innertopmargin=5pt,
leftmargin=0cm,
rightmargin=0cm,
innerbottommargin=5pt]{tBox}

% Exercise box	  
\newmdenv[skipabove=7pt,
skipbelow=7pt,
rightline=false,
leftline=true,
topline=false,
bottomline=false,
backgroundcolor=ocre!10,
linecolor=ocre,
innerleftmargin=5pt,
innerrightmargin=5pt,
innertopmargin=5pt,
innerbottommargin=5pt,
leftmargin=0cm,
rightmargin=0cm,
linewidth=4pt]{eBox}	

% Definition box
\newmdenv[skipabove=7pt,
skipbelow=7pt,
rightline=false,
leftline=true,
topline=false,
bottomline=false,
linecolor=ocre,
innerleftmargin=5pt,
innerrightmargin=5pt,
innertopmargin=0pt,
leftmargin=0cm,
rightmargin=0cm,
linewidth=4pt,
innerbottommargin=0pt]{dBox}	

% Corollary box
\newmdenv[skipabove=7pt,
skipbelow=7pt,
rightline=false,
leftline=true,
topline=false,
bottomline=false,
linecolor=gray,
backgroundcolor=black!5,
innerleftmargin=5pt,
innerrightmargin=5pt,
innertopmargin=5pt,
leftmargin=0cm,
rightmargin=0cm,
linewidth=4pt,
innerbottommargin=5pt]{cBox}

% Creates an environment for each type of theorem and assigns it a theorem text style from the "Theorem Styles" section above and a colored box from above
\newenvironment{theorem}{\begin{tBox}\begin{theoremeT}}{\end{theoremeT}\end{tBox}}
\newenvironment{exercise}{\begin{eBox}\begin{exerciseT}}{\hfill{\color{ocre}\tiny\ensuremath{\blacksquare}}\end{exerciseT}\end{eBox}}				  
\newenvironment{definition}{\begin{dBox}\begin{definitionT}}{\end{definitionT}\end{dBox}}	
\newenvironment{example}{\begin{exampleT}}{\hfill{\tiny\ensuremath{\blacksquare}}\end{exampleT}}		
\newenvironment{corollary}{\begin{cBox}\begin{corollaryT}}{\end{corollaryT}\end{cBox}}	

%----------------------------------------------------------------------------------------
%	REMARK ENVIRONMENT
%----------------------------------------------------------------------------------------

\newenvironment{remark}{\par\vspace{10pt}\small % Vertical white space above the remark and smaller font size
\begin{list}{}{
\leftmargin=35pt % Indentation on the left
\rightmargin=25pt}\item\ignorespaces % Indentation on the right
\makebox[-2.5pt]{\begin{tikzpicture}[overlay]
\node[draw=ocre!60,line width=1pt,circle,fill=ocre!25,font=\sffamily\bfseries,inner sep=2pt,outer sep=0pt] at (-15pt,0pt){\textcolor{ocre}{R}};\end{tikzpicture}} % Orange R in a circle
\advance\baselineskip -1pt}{\end{list}\vskip5pt} % Tighter line spacing and white space after remark

%----------------------------------------------------------------------------------------
%	SECTION NUMBERING IN THE MARGIN
%----------------------------------------------------------------------------------------

\makeatletter
\renewcommand{\@seccntformat}[1]{\llap{\textcolor{ocre}{\csname the#1\endcsname}\hspace{1em}}}                    
\renewcommand{\section}{\@startsection{section}{1}{\z@}
{-4ex \@plus -1ex \@minus -.4ex}
{1ex \@plus.2ex }
{\normalfont\large\sffamily\bfseries}}
\renewcommand{\subsection}{\@startsection {subsection}{2}{\z@}
{-3ex \@plus -0.1ex \@minus -.4ex}
{0.5ex \@plus.2ex }
{\normalfont\sffamily\bfseries}}
\renewcommand{\subsubsection}{\@startsection {subsubsection}{3}{\z@}
{-2ex \@plus -0.1ex \@minus -.2ex}
{.2ex \@plus.2ex }
{\normalfont\small\sffamily\bfseries}}                        
\renewcommand\paragraph{\@startsection{paragraph}{4}{\z@}
{-2ex \@plus-.2ex \@minus .2ex}
{.1ex}
{\normalfont\small\sffamily\bfseries}}

%----------------------------------------------------------------------------------------
%	HYPERLINKS IN THE DOCUMENTS
%----------------------------------------------------------------------------------------

% For an unclear reason, the package should be loaded now and not later
\usepackage{hyperref}
\hypersetup{hidelinks,backref=true,pagebackref=true,hyperindex=true,colorlinks=true,breaklinks=true,urlcolor=ocre,bookmarks=true,bookmarksopen=false,pdftitle={Title},pdfauthor={GEM Foundation}}

%----------------------------------------------------------------------------------------
%	CHAPTER HEADINGS
%----------------------------------------------------------------------------------------

% The set-up below should be (sadly) manually adapted to the overall margin page septup controlled by the geometry package loaded in the main.tex document. It is possible to implement below the dimensions used in the goemetry package (top,bottom,left,right)... TO BE DONE

\newcommand{\thechapterimage}{}
\newcommand{\chapterimage}[1]{\renewcommand{\thechapterimage}{#1}}

% Numbered chapters with mini tableofcontents
\def\thechapter{\arabic{chapter}}
\def\@makechapterhead#1{
\thispagestyle{empty}
{\centering \normalfont\sffamily
\ifnum \c@secnumdepth >\m@ne
\if@mainmatter
\startcontents
\begin{tikzpicture}[remember picture,overlay]
\node at (current page.north west)
{\begin{tikzpicture}[remember picture,overlay]
\node[anchor=north west,inner sep=0pt] at (0,0) {\includegraphics[width=\paperwidth]{\thechapterimage}};
%%%%%%%%%%%%%%%%%%%%%%%%%%%%%%%%%%%%%%%%%%%%%%%%%%%%%%%%%%%%%%%%%%%%%%%%%%%%%%%%%%%%%
% Commenting the 3 lines below removes the small contents box in the chapter heading
\fill[color=ocre!10!white,opacity=.6] (1cm,0) rectangle (8cm,-7cm);
\node[anchor=north west] at (1.1cm,.35cm) {\parbox[t][8cm][t]{6.5cm}{\huge\bfseries\flushleft \printcontents{l}{1}{\setcounter{tocdepth}{2}}}};
\draw[anchor=west] (5cm,-9cm) node [rounded corners=20pt,fill=ocre!10!white,text opacity=1,draw=ocre,draw opacity=1,line width=1.5pt,fill opacity=.6,inner sep=12pt]{\huge\sffamily\bfseries\textcolor{black}{\thechapter. #1\strut\makebox[22cm]{}}};
%%%%%%%%%%%%%%%%%%%%%%%%%%%%%%%%%%%%%%%%%%%%%%%%%%%%%%%%%%%%%%%%%%%%%%%%%%%%%%%%%%%%%
\end{tikzpicture}};
\end{tikzpicture}}
\par\vspace*{230\p@}
\fi
\fi}

% Unnumbered chapters without mini tableofcontents (could be added though) 
\def\@makeschapterhead#1{
\thispagestyle{empty}
{\centering \normalfont\sffamily
\ifnum \c@secnumdepth >\m@ne
\if@mainmatter
\begin{tikzpicture}[remember picture,overlay]
\node at (current page.north west)
{\begin{tikzpicture}[remember picture,overlay]
\node[anchor=north west,inner sep=0pt] at (0,0) {\includegraphics[width=\paperwidth]{\thechapterimage}};
\draw[anchor=west] (5cm,-9cm) node [rounded corners=20pt,fill=ocre!10!white,fill opacity=.6,inner sep=12pt,text opacity=1,draw=ocre,draw opacity=1,line width=1.5pt]{\huge\sffamily\bfseries\textcolor{black}{#1\strut\makebox[22cm]{}}};
\end{tikzpicture}};
\end{tikzpicture}}
\par\vspace*{230\p@}
\fi
\fi
}
\makeatother


%%%%%%%%%%%%%%%%%%%%%%%%%%%%%%%%%%%%%%%%%%%%%%%%%%%%%%%%%%%%%%%%%%%%%%%%%%
%           PYTHON ENVIRONMENT
%%%%%%%%%%%%%%%%%%%%%%%%%%%%%%%%%%%%%%%%%%%%%%%%%%%%%%%%%%%%%%%%%%%%%%%%%%
\definecolor{Code}{rgb}{0,0,0}
\definecolor{Decorators}{rgb}{0.5,0.5,0.5}
\definecolor{Numbers}{rgb}{0.5,0,0}
\definecolor{MatchingBrackets}{rgb}{0.25,0.5,0.5}
\definecolor{Keywords}{rgb}{0,0,1}
\definecolor{self}{rgb}{0,0,0}
\definecolor{Strings}{rgb}{0,0.63,0}
\definecolor{Comments}{rgb}{0,0.63,1}
\definecolor{Backquotes}{rgb}{0,0,0}
\definecolor{Classname}{rgb}{0,0,0}
\definecolor{FunctionName}{rgb}{0,0,0}
\definecolor{Operators}{rgb}{0,0,0}
\definecolor{Background}{rgb}{0.98,0.98,0.98}

\lstnewenvironment{python}[1][]{
\lstset{
numbers=left,
numberstyle=\footnotesize,
numbersep=1em,
xleftmargin=1em,
framextopmargin=2em,
framexbottommargin=2em,
showspaces=false,
showtabs=false,
showstringspaces=false,
frame=l,
tabsize=4,
% Basic
%basicstyle=\ttfamily\small\setstretch{1},
basicstyle=\ttfamily\footnotesize,
backgroundcolor=\color{Background},
language=Python,
% Comments
commentstyle=\color{Comments}\slshape,
% Strings
stringstyle=\color{Strings},
morecomment=[s][\color{Strings}]{"""}{"""},
morecomment=[s][\color{Strings}]{'''}{'''},
% keywords
morekeywords={import,from,class,def,for,while,if,is,in,elif,else,not,and,or,print,break,continue,return,True,False,None,access,as,,del,except,exec,finally,global,import,lambda,pass,print,raise,try,assert},
keywordstyle={\color{Keywords}\bfseries},
% additional keywords
morekeywords={[2]@invariant},
keywordstyle={[2]\color{Decorators}\slshape},
emph={self},
emphstyle={\color{self}\slshape},
%
}}{}
 


\begin{document}
\lstset{language=Python} % For listings environment - use python
% - - - - - - - - - - - - - - - - - - - - - - - - - - - - - -  Load the glossary
%\input{./book/glossary.tex}

%----------------------------------------------------------------------------------------
%	TITLE PAGE
%----------------------------------------------------------------------------------------

\begingroup
\thispagestyle{empty}
%\AddToShipoutPicture*{\put(6,5){\includegraphics[scale=1]{background}}} % Image background
\par\normalfont\fontsize{15}{15}\sffamily\selectfont
“OpenQuake: Calculate, share, explore”
\centering
\vspace*{9cm}
\par\normalfont\fontsize{35}{35}\sffamily\selectfont
Hazard Modeller's Toolkit - User Guide\par % Book title
\endgroup

%----------------------------------------------------------------------------------------
%	COPYRIGHT PAGE
%----------------------------------------------------------------------------------------

\newpage
~\vfill
\thispagestyle{empty}

\noindent Copyright \copyright\ 2014 GEM Foundation\\ % Copyright notice

\noindent \textsc{Published by GEM Foundation}\\ % Publisher

\noindent \textsc{globalquakemodel.org/openquake}\\ % URL

\noindent 
   {\textbf{Citation}} \hfill \\
   Please cite this document as:\\
   Weatherill, G. A. (2014) OpenQuake Hazard Modeller's Toolkit - User Guide. \textit{Global Earthquake Model (GEM). Technical Report}\\
   
   {\bf{Disclaimer}} \hfill \\
\noindent
   The ``Hazard Modeller's Tookit - User Guide'' is distributed in the hope that it will be useful, but without any warranty: without 
   even the implied warranty of merchantability or fitness for a 
   particular purpose. While every 
   precaution has been taken in the preparation of this document, in 
   no event shall the authors of the manual and the GEM Foundation be 
   liable to any party for direct, indirect, special, incidental, or 
   consequential damages, including lost profits, arising out of the 
   use of information contained in this document or from the use of 
   programs and source code that may accompany it, even if the authors 
   and GEM Foundation have been advised of the possibility of such damage. 
   The Book provided hereunder is on as "as is" basis, and the authors 
   and GEM Foundation have no obligations to provide maintenance, support,
   updates, enhancements, or modifications. 
   \hfill \\
   The current version of the book has been revised only by members of 
   the GEM model facility and it must be considered a draft copy. 
   %
   \vspace{0.4cm} \hfill \\
   {\bf{License}} \hfill \\
   This Book is distributed under the Creative Common License 
   Attribution-NonCommercial-NoDerivs 3.0 Unported (CC BY-NC-ND 3.0) 
   (see link below). You can download this Book and share it with 
   others as long as you provide proper credit, but you cannot change 
   it in any way or use it commercially. 
   \hfill \\

%\noindent \textit{First printing, June 2014} % Printing/edition date
\noindent \textit{Edition 2, January 2021} % Printing/edition date

%----------------------------------------------------------------------------------------
%	TABLE OF CONTENTS
%----------------------------------------------------------------------------------------

\chapterimage{./figures/chapter_head_1.pdf} % Table of contents heading image

\pagestyle{empty} % No headers

\tableofcontents % Print the table of contents itself

\cleardoublepage % Forces the first chapter to start on an odd page so it's on the right

\pagestyle{fancy} % Print headers again

%----------------------------------------------------------------------------------------
%	CHAPTER 1
%----------------------------------------------------------------------------------------
\chapterimage{./figures/chapter_head_2.pdf} % Chapter heading image
\chapter{Introduction}
\label{chap:intro}
The Hazard Modeller's Toolkit (or ``openquake.hmtk'') is a Python library of functions originally written by scientists at the GEM Model Facility, and now maintained by the GEM Foundation Secretariat. The HMTK is intended to provide 
scientists and engineers with the tools to help create the seismogenic 
input models that go into the OpenQuake hazard engine. The process of 
developing a hazard model is a complex and often challenging 
one, and while many aspects of the practice are relatively common, the 
choice of certain methods or tools for undertaking each step can be a 
matter of judgement. The intention of this software is to provide 
scientists and engineers with the means to apply many of the most 
commonly used algorithms for preparing seismogenic source models 
using seismicitiy and geological data. 

This manual is Version 2.0 of the HMTK tutorial. The major differences in the toolkit and the tutorial compared to the original release are i) the HMTK is now contained in the OpenQuake Engine, and does not require any separate installation, ii) the OpenQuake $hazardlib$ source classes have been adopted in order to ensure full compatibility and consistency between the two libraries, and iii) the plotting functions that produce maps now use Generic Mapping Tools (GMT) and Python scripts housed in the OpenQuake Model Building Toolkit. 


\section{The Development Process}

The Hazard Modeller's Toolkit is developed by GEM, and has occurred in several different stages. The present version makes the modelling tools available as a library, reflecting the general trend in the OpenQuake development process toward having a modular software framework. This means that the modelling - hazard - risk process is separated into libraries (e.g. oq-hazardlib, oq-risklib) that can be utilised as standalone tools, in addition to being integrated within the OpenQuake engine and platform. This is designed to allow for flexibility in the process, and also allow the user to begin to utilise (possibly in other contexts) functions and classes that are intended to address particular stages of the calculation. Such an approach ensures that each sub-component of the toolkit is fully tested, with a minimal degree of duplication in the testing process. In the HMTK this is taken a step further, as we are aiming to provide the hazard modeller as much control over the modelling process as possible, while retaining as complete a level of code testing as is practical to implement given the development resources available. 

The HMTK aims to address particular objectives:

\begin{description}
\item[Portability] Reduction in the number of Python dependencies to allow for a high degree of cross-platform deployment 

\item[Adaptability] Cleaner separation of methods into self-contained components that can be implemented and tested within requiring adaption of the remainder of the code.

\item[Abstraction] This concept is often a critical component object-oriented development. It describes the specification of a core behaviour of a method, which implementations (by means of the subclass) must follow. For example, a declustering algorithm must follow the common behaviour path, in this instance i) reading and earthquake catalogue and some configurable parameters, ii) identifying the clusters of events, iii) identifying the mainshocks from within each cluster,iv) returning this information to the user. The details of the implementation are then dependent on the algorithm, providing that the core flow is met. This is designed to allow the algorithms to be \emph{interchangeable} in the sense that different methods for  particular task could be selected with no (or at least minimal) modification to the rest of the code.

\item[Usability] The creation of a library which could itself be embedded within larger applications (e.g. as part of a graphical user interface).
 
\end{description}



\section{Getting Started and Running the Software}

The Modeller's Toolkit and associated software are designed for execution 
from the command line. As with the OpenQuake Engine, the preferred environment is 
Ubuntu Linux (12.04 or later), but is also supported on other operating systems.
Since the HMTK is contained by the OpenQuake Engine, all dependencies required by the HMTK itself are installed alongside the OpenQuake Engine. For more information regarding the current dependencies and installing the OpenQuake Engine, see \href{https://github.com/gem/oq-engine}{https://github.com/gem/oq-engine}.


\subsection{Current Features}

The Hazard Modeller's Toolkit is currently divided into three sections: 

\begin{enumerate}
\item \textbf{Earthquake Catalogue and Seismicity Analysis}
    These functions are intended to address the needs of defining seismic activity rate from an earthquake catalogue. They algorithms for identification of Non-Poissonian events (declustering), analysis of catalogue completeness, calculation of activity rate and b-value and, finally, estimation of maximum magnitude using statistical analyses of the earthquake catalogue. Also included in these tools is an initial implementation of a smoothed seismicity algorithm using the \textcite{frankel1995} approach.
     
\item \textbf{Active Faults Source Models from Geological Data}

    These functions are intended to address the Modeller needs for defining earthquake activity rates on fault sources from the geological slip rate, including support for some epistemic uncertainty analysis on critical parameters in the process.

\item \textbf{Seismic Source Models from Geodetic Data}

    These functions are intended to address the use of geodetic data to derive seismic activity rates from a strain rate model for a region, implementing the Seismic Hazard Inferred from Tectonics (SHIFT) methodology developed by \textcite{BirdLiu2007} and applied on a global scale by \textcite{Bird_etal2010}.
\end{enumerate}

A summary of the algorithms available in the present version is given in Table \ref{tab:current_features}.
\begin{table}
\centering
\begin{tabular}{|c|c|} \hline
\textbf{Feature} & \textbf{Algorithm}\\ \hline
\textbf{Seismicity} & \\ \hline
Declustering & \textcite{GardnerKnopoff1974}  \\
    & AFTERAN \parencite{Musson1999} \\ \hline
Completeness & \textcite{Stepp1971}\\ \hline
Recurrence & Maximum Likelihood \parencite{Aki1965}\\
 & Time-dependent MLE\\
 & \textcite{Weichert1980}\\ \hline
 Smoothed Seismicity & \textcite{frankel1995} \\ \hline
 \textbf{Geology} & \\ \hline
 Recurrence & \textcite{AndersonLuco1983} ``Arbitrary''\\
  & \textcite{AndersonLuco1983} ``Area $M_{MAX}$''\\
  & Characteristic (Truncated Gaussian) \\
  & \textcite{YoungsCoppersmith1985} Exponential\\
  & \textcite{YoungsCoppersmith1985} Characteristic\\ \hline
 \textbf{Geodetic Strain} & \\ \hline
 Recurrence & Seismic Hazard Inferred from Tectonics (SHIFT) \\
           &  \textcite{BirdLiu2007, Bird_etal2010} \\ \hline
\end{tabular}
\caption{Current algorithms in the HMTK}
\label{tab:current_features}
\end{table}

\subsection{About this Tutorial}

As previously indicated, the Modeller's Toolkit itself is a Python library. This means that its functions can be utilised in many different python applications. It is not, at present, a stand-alone software, and requires some investment of time from the user to understand the functionalities and learn how to link the various tools together into a workflow that will be suitable for the modelling problem at hand.

This manual is designed to explain the various functions in the toolkit and to provide some illustrative examples showing how to implement them for particular contexts and applications. The tutorial itself does not specifically require a working knowledge of Python. However, an understanding of the basic python data types, and ideally some familiarity with the use of Python objects, is highly desirable. Users who are new to Python are recommended to familiarise themselves with Appendix \ref{sec:python_guide} of this tutorial. This provides a brief overview of the Python programming language and should introduce concepts such as classes and dictionaries, which will be encountered in due course. For more detail of the complete Python language, a comprehensive overview of its features and usage standard python documentation (\href{http://docs.python.org/2/tutorial/}{http://docs.python.org/2/tutorial/}). Where necessary particular Python programming concepts will be explained in further detail.

The code snippets (indicated by verbatim text) can be executed from within an ''Interactive Python (IPython)'' environment, or may form the basis for usage of the openquake.hmtk in other python scripts that the user may wish to run construct themselves. If not already installed on your system, IPython can be installed from the python package repository by entering: 

\begin{Verbatim}[frame=single, commandchars=\\\{\}, fontsize=\scriptsize]
~\$ sudo pip install ipython
\end{Verbatim}

An ``interactive'' session can then be opened by typing \verb=ipython= at the command prompt. If \verb=matplotlib= is installed and you wish to use the plotting functionalities described herein then you should open IPython with the command:

\begin{Verbatim}[frame=single, commandchars=\\\{\}, fontsize=\scriptsize]
~\$ ipython --pylab
\end{Verbatim}

To exit an IPython session at any time simply type \verb=exit=.

For a more visual application of the openquake.hmtk the reader is encouraged to utilise the ``IPython Notebook'' (\href{http://ipython.org/notebook.html}{http://ipython.org/notebook.html}). This novel tool implements IPython inside a web-browser environment, permitting the user to create and store real Python workflows that can be retrieved and executed, whilst allowing for images and text to be embedded. A screenshot of the openquake.hmtk used in an IPython Notebook environment is shown in Figure \ref{fig:notebook}. From version 1.0 of IPython, the IPython Notebook comes installed. A notebook session can be
started via the command:

\begin{Verbatim}[frame=single, commandchars=\\\{\}, fontsize=\scriptsize]
~\$ ipython notebook --pylab inline
\end{Verbatim}

\begin{figure}[htb]
  \centering
      \includegraphics[width=\textwidth]{./figures_v2/hmtk_notebook_screenshot.jpg}
  \caption{Example of the openquake.hmtk embedded in an IPython Notebook}
  \label{fig:notebook}
\end{figure}


\subsection{Visualisation}

In addition to the scientific tools, which will be described in detail in due course, the original version of the openquake.hmtk also included a set of functionalities for visualisation of data and results pertinent to the preparation of seismic hazard input models. While not considered an essential component of the openquake.hmtk, the usage of the plotting functions can facilitate model development. Particular visualisation functions shall be referred to where relevant for the particular tool or data set. 

The current version of the HMTK includes most of the original visualisation tools. However, the tools for map creation have been depracated, and replaced by a set of mapping functions in the \href{https://github.com/GEMScienceTools/oq-mbtk/tree/master/openquake}{OpenQuake Model Building Toolkit (MBTK)}. The tools now use \href{https://www.generic-mapping-tools.org}{Generic Mapping Tools (GMT)}, and so were moved outside of the HMTK library as to not add GMT as a dependency of the OpenQuake Engine. However, the mapping functions are still described in this tutorial in order to provide users with a replacement to the depracated functions. 

\subsubsection{Mapping tools: additional setup}
Within the plotting tools is a set of methods to create maps of geospatial data; these tools are housed in the MBTK. Use of the mapping tools requires the following additional package installations:\\

\begin{enumerate}
	\item \href{https://www.generic-mapping-tools.org/download/}{GMT} version 6.0 or later.
	\item \href{https://github.com/GEMScienceTools/oq-mbtk/tree/master/openquake}{MBTK}\\
\end{enumerate}

The functions of \cprotect{\href{https://github.com/GEMScienceTools/oq-mbtk/tree/master/openquake/plt}}{\verb=openquake.plt.mapping=} can also be used independently of the MBTK (NB: the descriptions herein assume that mapping functions are called through the MBTK). 

\subsubsection{Map Creation}
An IPython Notebook demonstrating how to use the mapping methods can be found \href{https://github.com/GEMScienceTools/oq-mbtk/tree/master/openquake/plt/demo}{in the MBTK}. The basic functionalities are described herein.

To set-up a simple basemap it is necessary to define the configuration of the plot (such as spatial limit and coastline resolution). This is done as follows:

\begin{python}[frame=single]
In [1]: from openquake.plt.mapping import HMTKBaseMap

In [2]: map_config = {"min_lon": 18.0,
                      "max_lon": 32.0,
                       "min_lat": 33.0,
                       "max_lat": 43.0,
		       "title": "Title of Map"}

In [3]: basemap1 = HMTKBaseMap(map_config)
\end{python}

\verb=HMTKBaseMap= is instantiated with a dictionary of configuration parameters: minimum longitude (\verb=min_lon=), maximum longitude (\verb=max_lon=), minimum latitude (\verb=min_lat=), maximum latitude (\verb=max_lat=). The map title (\verb=title=) can also be specified.

A few other configurations can be passed to \verb=HMTKBaseMap= via keyword parameters during the map instantiation:

\begin{itemize}
	\item \verb=projection=: String beginning with '-J' that indicates the map projection, and optionally central meridian and scaling, following the GMT syntax (\href{http://gmt.soest.hawaii.edu/doc/latest/gmt.html\#j-full}{GMT Map Projections}). The default `-JM15c' is a Mercator projection 15 cm wide.
\item \verb=lat_lon_spacing=: Indicates the spacing of latitude and longitude tickmarks. The default is 2 degrees.
\item \verb=output_folder=: Denotes the output directory for the final map and associated files (if saved, see \verb=.savemap=) relative to the local path. The \verb=output_folder= is immeidately created by \verb=HMTKBaseMap=, and used for all temporary files created during the mapping process. The default is \textit{gmt}. 
\item \verb=overwrite=: If True, gives permission to overwrite all existing files in the specified \verb=output_folder=.\\
\end{itemize}

\noindent The class \verb=HMTKBaseMap= contains a set of methods for mapping catalogue data or simplified source models:\\

\noindent \verb;.add_catalogue(cat, scale=0.05, cpt_file=`tmp.cpt', color_field=`depth',;\\
		\verb;logscale=True);\\

\noindent This function will overlay an earthquake catalogue onto the basemap. The input value \verb=cat= is the earthquake catalogue as an instance of the class \\\verb=openquake.hmtk.seismicity.catalogue.Catalogue= (see the next section for details). The catalogue is the only mandatory parameter, but the user can also specify the following optional parameters:\\

\begin{itemize}
	\item \verb=scale=: a scaling coefficient that sets the symbol size per magnitude $m$. Size follows the equation ${\verb=scale=}*10^{(-1.5+m*0.3)}$, where m is magnitude. See GMT documentation.
	\item \verb=cpt_file=: name of an existing color pallet to color earthquake markers. If not specified, the default "tmp.cpt" is generated based on the catalogue \verb=color_field=
	\item \verb=color_field=: the parameter used to color the earthquake markers. The given field must correspond to the catalogue header. If not specified, the markers are colored by depth.
	\item \verb=logscale=: if `True', generates the color pallet according to a log scale. `False' uses a linear color scale. Default is `True'. Ignored if \verb=cpt_file= is specified.\\
\end{itemize}


\begin{figure}[htb]
  \centering
      \includegraphics[width=\textwidth]{./figures_v2/catalogue.jpg}
      %\includegraphics[trim=20mm 14mm 1mm 1mm, clip, width=\textwidth]{./figures_v2/catalogue.jpg}
  \caption{Example visualisation of an Earthquake Catalogue}
  \label{fig:eqcat_simple}
\end{figure}


\noindent \verb;.add_source_model(model);\\

\noindent This method adds a source model to the basemap. The input value \verb=model= is an instance of the class \verb=openquake.hazardlib.nrml.SourceModel= (see section \textbf{TODO -> this is replacing the mtk source classes}). An example of a source model plot is shown in \ref{fig:source_model_map}.

NB: At present, only the following source typologies can be plotted automatically:

\begin{itemize}
\item Point sources
\item Simple faults
\item Complex faults
\item Area sources\\
\end{itemize}

Non-parametric sources and multi-point sources will be added soon.\\
 
\begin{figure}[htb]
  \centering
      \includegraphics[width=\textwidth]{./figures_v2/PNGSourceModel.jpg}
	\caption{Example visualisation of a source model for Papua New Guinea \textcite{ghasemi2016} with area sources (blue) and a complex fault.}
  \label{fig:source_model_map}
\end{figure}


\noindent \verb;.add_colour_scaled_points(longitude, latitude, data, label=`', shape=`-Ss', ;\\
\verb;                                    size=0.3, logscale=False);\\

\noindent This method overlays a set of data points with colour scaled according to the \verb=data= values. Three data arrays are required: one each with the \verb=longitude= and \verb=latitude= coordinates of the data points, and \verb=data=, a set of scalar values (e.g. magnitude or depth, if plotting an earthquake catalogue) associated with those points. In addition to these, the method takes four optional keyword parameters:\\
\begin{itemize}
\item \verb=label=: a string used to label the color scale; corresponds to \verb=data=
\item \verb=shape=: a string indicating the shape of the data markers, using GMT syntax starting with `-S' (see \href{https://docs.generic-mapping-tools.org/latest/psxy.html\#s}{GMT psxy markers}. The default, `-Ss' is squares.
\item \verb=size=: the size in cm of the plotted markers (see \href{https://docs.generic-mapping-tools.org/latest/psxy.html\#s}{GMT psxy markers}). Default is 0.3 cm.
\item \verb=logscale=: if True, use a logscale to create the colorbar. Default is False.\\
\end{itemize}

\begin{figure}[htb]
  \centering
      \includegraphics[width=\textwidth]{./figures_v2/colorscaled.jpg}
	\caption{Example of a seismicity catalogue with color scaled by magnitude.}
  \label{fig:cat_color_scaled}
\end{figure}


\noindent \verb;.add_size_scaled_points(longitude, latitude, data, shape=`-Ss',;\\
\verb;                       logplot=False, color=`blue', smin=0.01, coeff=1.0, ;\\
\verb;                       sscale=2.0, label=`', legend=True);\\

\noindent This method overlays a set of data points with size scaled according to the \verb=data= values. Three data arrays are required: one each with the \verb=longitude= and \verb=latitude= coordinates of points to be plotted, and \verb=data=, a set of scalar values associated with those points. In addition to these, the method takes eight optional keyword parameters:\\

\begin{itemize}
\item \verb=shape=: a string indicating the shape of the data markers, using GMT syntax starting with `-S' (see \href{https://docs.generic-mapping-tools.org/latest/psxy.html\#s}{GMT psxy markers}. The default, `-Ss' is squares.
\item \verb=logplot=: if True, use a logscale to create the marker sizes. Default is False.
\item \verb=color=: a string that indicates the marker color (see \href{https://docs.generic-mapping-tools.org/latest/psxy.html\#w}{GMT psxy markers}. Default is `blue'.
\item \verb=smin=: size of the smallest symbol in cm. Marker size is computed as\\ ${\verb=smin=}+{\verb=coeff=}\times {\verb=data=}^{\verb=sscale=}$. Default is 0.01 cm.
\item \verb=coeff=: used with \verb=sscale= and \verb=smin= to set the marker sizes. Default is 1.0.
\item \verb=sscale=: used with \verb=coeff= and \verb=smin= to set the marker sizes. Default is 2.0.
\item \verb=label=: a string that corresponds to the \verb=data= array
\item \verb=legend=: if True, adds a legend to the plot. Default is True. \\
\end{itemize}

\begin{figure}[htb]
  \centering
      \includegraphics[width=\textwidth]{./figures_v2/sizescaled.jpg}
	\caption{Example of a seismicity catalogue with size scaled by magnitude.}
  \label{fig:source_model_map}
\end{figure}

\noindent \verb;.add_focal_mechanism(filename, mech_format);\\

\noindent This method overlays focal mechanisms. The string \verb=filename= indicates a file containing focal mechanism data. \verb=mech_format= is a string contained by quotations used to indicate the data format used by \verb=filename=, allowing two options---focal mechanism (`FM') and seismic moment tensor (`MT')---both using the Harvard CMT convention, as described by \href{https://docs.generic-mapping-tools.org/latest/supplements/seis/psmeca.html?highlight=psmeca\#s}{GMT psmeca}.\\

\noindent \verb;.savemap(filename=None, save_script=False, verb=False);\\

An instance of \verb=HMTKBaseMap= is not automatically saved. In order to do so, the method \verb=.savemap()= must be called, finalizing and executing the GMT script. The method can take the following three keyword arguments:\\

\begin{itemize}
\item \verb=filename=: a string used to name the map, which includes a suffix (limited to `.pdf', `.png', and `.jpg') indicating the desired file type. If not specified, the map is saved as $map.pdf$ in the directory \verb=output_folder= that was assigned during the \verb=HMTKBaseMap= instantiatation.
\item \verb=save_script=: if True, the GMT commands are saved to a shell script, and this with all files needed to create the map are saved in \verb=output_folder=. If False (the default), all the temporary files are erased and only the map is saved.
\item \verb=verb= (verbose): if True, GMT commands are printed as they are executed.\\
\end{itemize}

The \verb=save_script= option gives the user more flexibility to modify the plot settings than are available through the methods, while providing the structure of the GMT script as a starting point. NB: Take care not to overwrite scripts that have been customized by rerunning the mapping code! \\

The \verb=.savemap()= method is used as follows (continuing from the above Python lines):\\

\begin{python}[frame=single]
In [4]: finame = 'map_demo.pdf'

In [5]: basemap1.savemap(filename=finame, save_script=True)
\end{python}



%----------------------------------------------------------------------------------------
%	CHAPTER 2
%----------------------------------------------------------------------------------------
\chapterimage{./figures/chapter_head_2.pdf} % Chapter heading image
\chapter{Catalogue Tools}
\label{chap:catalogue}
\section{The Earthquake Catalogue}

The seismicity tools are intended for use in deriving activity rates from an observed earthquake catalogue, which may include both instrumental and historical seismicity. The tools are broken down into five separate libraries: i) Declustering, ii) Completeness, iii) Calculation of Gutenberg-Richter a- and b-value, iv) Statistical estimators of maximum magnitude from seismicity) and v) Smoothed Seismicity. In a common use case it is likely that many of the above methods, particularly recurrence and maximum magnitude estimation, may need to be applied to a selected sub-catalogue (e.g. earthquakes within a particular polygon). The toolkit allows for the creation of a source model containing one or more of the supported OpenQuake seismogenic source typologies, which can be used as a reference for selection, e.g. events within an area source (polygon), events within a distance of a fault etc. The supported input formats for both the catalogue are described below, and the source models in the subsequent chapter. 

\subsection{The Catalogue Format and Class}

The input catalogue must be formatted as a comma-separated value file (.csv), with the following attributes in the header line (attributes with an * indicate essential attributes), although the order of the columns need not be fixed:

\begin{table}
\begin{tabular}{|l|l|}  \hline 
Attribute & Description \\ \hline
eventID* & A unique identifier (integer) for each earthquake in the catalogue \\
Agency & The code (string) of the recording agency for the event solution  \\
year* & Year of event (integer) in the range -10000 to present \\
 & (events before common era (BCE) should have a negative value)\\
month* & Month of event (integer)\\
day* & Day of event (integer) \\
hour* & Hour of event (integer) - if unknown then set to 0 \\
minute* & Minute of event (integer) - if unknown then set to 0 \\
second* & Second of event (float) - if unknown set to 0.0 \\
timeError & Error in event time (float) \\
longitude* & Longitude of event, in decimal degrees (float) \\
latitude* & Latitude of event, in decimal degrees (float) \\
SemiMajor90 & Length (km) of the semi-major axis of the 90 \% \\
            & confidence ellipsoid for location error (float) \\
SemiMinor90 & Length (km) of the semi-minor axis of the 90 \% \\
            & confidence ellipsoid for location error (float) \\
ErrorStrike & Azimuth (in degrees) of the 90 \% \\
            & confidence ellipsoid for location error (float) \\
depth* & Depth (km) of earthquake (float)\\
depthError & Uncertainty (as standard deviation) in earthquake depth (km) (float)\\
magnitude* & Homogenised magnitude of the event (float) - typically Mw \\
sigmaMagnitude* & Uncertainty on the homogenised magnitude (float) typically Mw \\ \hline
\end{tabular}
\caption{List of Attributes in the Earthquake Catalogue File (* Indicates Essential)}
\label{tab: EQCatalogueFormat}
\end{table}

To load the catalogue using the IPython environment, in an open IPython session type:

%\begin{Verbatim}[frame=single, commandchars=\\\{\}, fontsize=\scriptsize, samepage=true]
\begin{python}
>> from openquake.hmtk.parsers.catalogue import CsvCatalogueParser
>> catalogue_filename = 'path/to/catalogue_file.csv'
>> parser = CsvCatalogueParser(catalogue_filename)
>> catalogue = parser.read_file()
\end{python}

\textbf{N.B. the csv file can contain additional attributes of the catalogue too and will be parsed correctly; however, if the attribute is not one that is specifically recognised by the catalogue class then a message will be displayed indicating:}

\begin{Verbatim}[frame=single, commandchars=\\\{\}, fontsize=\scriptsize, samepage=true]
Catalogue Attribute ... is not a recognised catalogue key 
\end{Verbatim}

\textbf{This is expected behaviour and simply indicates that although this data is given in the input file, it is not retained in the data dictionary.}

The variable \verb=catalogue= is an instance of the class openquake.hmtk.seismicity.catalogue.Catalogue, which now contains the catalogue itself (as \verb=catalogue.data=) and some methods that can be applied to the catalogue. The first attribute (\verb=catalogue.data=), is a dictionary where each attribute of the catalogue is either a 1-D numpy vector (for float and integer values) or a python list (for string values). For example, to return a vector containing all the magnitudes in the \verb=magnitude= column of the catalogue simply type:

\begin{python}
>> catalogue.data['magnitude']
array([ 6.5,  6.5,  6. , ...,  4.8,  5.2,  4.1])
\end{python}

The catalogue class contains several helpful methods (called via \verb=catalogue. ...=):
\begin{itemize}
\item \verb=catalogue.get_number_events()= Returns the number of events currently in the catalogue (integer)

\item \verb=catalogue.load_to_array(keys)= Returns a numpy array of floating data, with the columns ordered according to the list of keys. If the key corresponds to a string item (e.g. Agency) then an error will be raised.

\begin{python}[frame=single]
>> catalogue.load_to_array(['year', 'longitude', 'latitude',
                            'depth', 'magnitude'])
array([[ 1910. ,  26.941 ,  38.507 ,  13.2 ,  6.5 ],
       [ 1910. ,  22.190 ,  37.720 ,  20.4 ,  6.5 ],
       [ 1910. ,  28.881 ,  33.274 ,  25.0 ,  6.0 ],
       ..., 
       [ 2009. ,  20.054 ,  39.854 ,  20.2 ,  4.8 ],
       [ 2009. ,  23.481 ,  38.050 ,  15.2 ,  5.2 ],
       [ 2009. ,  28.959 ,  34.664 ,  18.4 ,  4.1 ]]) 
\end{python}

\item \verb=catalogue.load_from_array(keys, data_array)= Creates the catalogue data dictionary from an array, given header as an ordered list of dictionary keys. This can be used in the case where the earthquake catalogue is loaded in a simple ascii format. For example, if the user wishes to load in a catalogue from the Zmap format, which gives the columns as:

\begin{verbatim}
longitude, latitude, year, month, day, magnitude, depth, hour, 
minute, second
\end{verbatim}

This file type could be parsed into a catalogue without the need of a specific parser, as follows:

\begin{python}[frame=single]
>> import numpy
# Assuming no headers in the file 
# (set skip_header=1 if headers are found)
>> data = numpy.genfromtxt('PATH/TO/ZMAP_FILE.txt',
                           skip_header=0)

>> headers = ['longitude', 'latitude', 'year', 'month',
              'day', 'magnitude', 'depth', 'hour', 
              'minute', 'second']

# Create instance of a catalogue class
>> from openquake.hmtk.seismicity.catalogue import Catalogue
>> catalogue = Catalogue()

# Load the data array into the catalogue
>> catalogue.load_from_array(data, headers)
\end{python}
 

\item \verb=catalogue.get_decimal_time()= 

Returns the time of the earthquake in a decimal format

\item \verb=catalogue.hypocentres_as_mesh()=

Returns the hypocentres of an earthquake as an instance of the class \\ ``openquake.hazardlib.geo.mesh.Mesh'' (useful for geospatial functions)

\item \verb=catalogue.hypocentres_to_cartesian()=

Returns the hypocentres in a 3D cartesian framework

\item \verb=catalogue.purge_catalogue(flag_vector)=

Purges the catalogue of all \verb=False= events in the boolean vector. Thus is used for removing foreshocks and aftershocks from a catalogue after the application of a declustering algorithm.

\item \verb=catalogue.sort_catalogue_chronologically()=

Sorts an input into chronological order. \\
\emph{N.B. Some methods will implicitly assume that the catalogue is in chronological order, so it is recommended to run this function if you believe that there may be events out of order}

\item \verb=catalogue.select_catalogue_events(IDX)=

Orders the catalogue according to the event order specified in IDX. Behaves the same as \verb=purge_catalogue(IDX)= if IDX is a boolean vector

\item \verb;catalogue.get_depth_distribution(depth_bins, normalisation=False,;\\
\verb;    bootstrap=None);

Returns a depth histogram for the catalogue using bins specified by \verb=depth_bins=. If \verb;normalisation=True; then the function will return the histogram as a probability mass function, otherwise the original count will be returned. If uncertainties are reported on depth such that one or more values in \\ \verb=catalogue.data['depthError']= are greater than 0., the function will perform a bootstrap analysis, taking into account the depth error, with the number of bootstraps given by the keyword \verb=bootstrap=. 

\begin{python}[frame=single]
# Import numpy and matplotlib
>> import numpy as np
>> import matplotlib.pyplot as plt

# Define depth bins for (e.g) 
# 0. - 150 km in intervals of 10 km
>> depth_bins = np.arange(0., 160., 10.)

# Get normalised histograms (without bootstrapping)
>> depth_hist = catalogue.get_depth_distribution(
    depth_bins,
    normalisation=True)
\end{python}

To generate a simple histogram plot of hypocentral depth, the process below can be followed to produce a depth histogram similar to the one shown in Figure \ref{fig:simple_depth_hist}:

\begin{python}[frame=single]
>> from openquake.hmtk.plotting.seismicity.catalogue_plots import\
    plot_depth_histogram

>> depth_bin = 5.0
>> plot_depth_histogram(catalogue,
                        depth_bin,
                        filename="/path/to/image.eps",
                        filetype="eps")
\end{python}
\begin{figure}[htb]
  \centering
      \includegraphics[trim=10mm 8mm 10mm 10mm, clip, width=12cm]{./figures/simple_depth_histogram.eps}
  \caption{Example depth histogram}
  \label{fig:simple_depth_hist}
\end{figure}

\item \verb;catalogue.get_magnitude_depth_distribution(magnitude_bins, depth_bins,;\\
\verb;    normalisation=False, bootstrap=None); 

Returns a two-dimensional histogram of magnitude and hypocentral depth, with the corresponding bins defined by the vectors \verb=magnitude_bins= and \verb=depth_bins=. The options \verb=normalisation= and \verb=bootstrap= are the same as for the one dimensional histogram. The usage is illustrated below:

\begin{python}[frame=single]]
# Define depth bins for (e.g) 
# 0. - 150 km in intervals of 55 km
>> depth_bins = np.arange(0., 155., 5.)

# Define magnitude bins (e.g.) 2.5 - 7.6 in intervals of 0.1
>> magnitude_bins = np.arange(2.5, 7.7, 0.1)

# Get normalised histograms (without bootstrapping)
>> m_d_hist = catalogue.get_magnitude_depth_distribution(
    magnitude_bins,
    depth_bins,
    normalisation=True,
    bootstrap=None)
\end{python}

To generate a plot of magnitude-depth density, the following function can be used to produce a figure similar to that shown in Figure \ref{fig:mag_depth_density}.

\begin{python}[frame=single]
>> from openquake.hmtk.plotting.seismicity.catalogue_plots import\
     plot_magnitude_depth_density
>> magnitude_bin = 0.1
>> depth_bin = 5.0 
>> plot_magnitude_depth_density(
    catalogue,
    magnitude_bin
    depth_bin,
    logscale=True, \# Logarithmic colour scale
    filename="/path/to/image.eps", \# Optional
    filetype="eps")   \# Optional
\end{python}

\begin{figure}[htb]
  \centering
      \includegraphics[trim=10mm 10mm 10mm 10mm, clip, width=14cm]{./figures/magnitude_depth_density.eps}
  \caption{Example magnitude-depth density plot}
  \label{fig:mag_depth_density}
\end{figure}

\item \verb;catalogue.get_magnitude_time_distribution(magnitude_bins, time_bins,;\\
\verb; normalisation=False, bootstrap=None);

Returns a 2D histogram of magnitude with time. \verb=time_bins= are the bin edges for the time windows, in decimal years. To run the function simple follow:

\begin{python}[frame=single]
# Define annual time bins from 1900 CE to 2012 CE
>> time_bins = np.arange(1900., 2013., 1.)
# Define magnitude bins (e.g.) 2.5 - 7.6 in intervals of 0.1
>> magnitude_bins = np.arange(2.5, 7.7, 0.1)
# Get normalised histograms (without bootstrapping)
>> mag_time_hist = catalogue.get_magnitude_time_distribution(
    magnitude_bins,
    time_bins,
    normalisation=True,
    bootstrap=None)
\end{python}

To automatically generate a plot, similar to that shown in Figure \ref{fig:mag_time_density} , run the following:

\begin{python}[frame=single]
>> from openquake.hmtk.plotting.seismicity.catalogue_plots import\
    plot_magnitude_time_density
>> magnitude_bin_width = 0.1
>> time_bin_width = 0.1 
>> plot_magnitude_time_density(catalogue,
                               magnitude_bin_width,
                               time_bin_width,
                               filename="/path/to/image.eps", 
                               filetype="eps")
\end{python}

\begin{figure}[htb]
  \centering
      \includegraphics[trim=10mm 8mm 10mm 10mm, clip, width=12cm]{./figures/magnitude_time_density.eps}
  \caption{Example magnitude-time density plot}
  \label{fig:mag_time_density}
\end{figure}
\end{itemize}

\subsection{The ``Selector'' Class}

In the process of constructing a PSHA seismogenic source model from seismicity it is necessary to select sub-sets of the earthquake catalogue, usually for calculating earthquake recurrence statistics pertinent to a particular region or seismogenic source. As catalogue selection is such a prevalent aspect of the source modelling process, the selection is done inside the HMTK via the use of a "Selector" tool. This tools is a container for all methods associated with the selection of sub-catalogues from a given earthquake catalogue. It will be seen in due course that later methods relating to the selection of the catalogue for a particular source require as an input an instance of the selector class, rather than the catalogue itself.

To setup the ``Selector'' tool:

\begin{python}[frame=single]
>> from openquake.hmtk.seismicity.selector import CatalogueSelector

# Assuming that there already exists a 
# catalogue named ''catalogue1''

>> selector1 = CatalogueSelector(catalogue1,
                                 create_copy=True)

\end{python}

The optional keyword \verb=create_copy= ensures that when the events not selected are purged from the catalogue a ``deepcopy'' is taken of the original catalogue. This ensures that the original catalogue remains unmodified when a subset of events is selected.

The catalogue selector class has the following methods:

\verb;.within_polygon(polygon, distance=None);

Selects events within a polygon described by the class \verb=openquake.hazardlib.geo.=\\\verb=polygon.Polygon=. \verb=distance= is the distance (in km) to use as a buffer, if required. Optional keyword arguments \verb=upper_depth= and \verb=lower_depth= can be used to limit the depth range of the catalogue returned by the selector to only those events whose hypocentres are within the specified depth limits.

\verb;.circular_distance_from_point(point, distance, distance_type="epicentral");

Selects events within a distance from the a location. The location (\verb=point=) is an instance of the openquake.hazardlib.geo.point.Point class, whilst \verb=distance= is the selection distance (km) and \verb=distance_type= can be either "epicentral" or "hypocentral".  

\verb;.cartesian_square_centred_on_point(point, distance);

Selects events within a square of side length \verb=distance=, on a location (represented as an openquake \verb=Point= class).

\verb;.within_joyner_boore_distance(surface, distance);

Returns earthquakes within a distance (km) of the surface projection (``Joyner-Boore'' distance) of a fault surface. The fault surface must be defined as an instance of the class \\\verb=openquake.hazardlib.geo.surface.simple_fault.SimpleFaultSurface= or\\ \verb=openquake.hazardlib.geo.surface.complex_fault.ComplexFaultSurface=.

\verb;.within_rupture_distance(surface, distance);

Returns earthquakes within a distance (km) of a fault surface. The fault surface must be defined as an instance of the class \\\verb=openquake.hazardlib.geo.surface.simple_fault.SimpleFaultSurface= or\\ \verb=openquake.hazardlib.geo.surface.complex_fault.ComplexFaultSurface=.

\verb;.within_time_period(start_time=None, end_time=None);

Selects earthquakes within a time period. Times must be input as instances of a \verb=datetime= object. For example:

\begin{python}[frame=single]
>> from datetime import datetime
>> selector1 = CatalogueSelector(catalogue1, create_copy=True)
# Early time limit is 1 January 1990 00:00:00
>> early = datetime(1990, 1, 1, 0, 0, 0)
# Late time limit is 31 December 1999 23:59:59
>>: late = datetime(1999, 12, 31, 23, 59, 59)
>> catalogue_nineties = selector1.within_time_period(
    start_time=early,
    end_time=late)
\end{python}

\verb;.within_depth_range(lower_depth=None, upper_depth=None);

Selects earthquakes whose hypocentres are within the range specified by the lower depth limit (\verb=lower_depth=) and the upper depth limit (\verb=upper_depth=), both in km.

\verb;.within_magnitude_range(lower_mag=None, upper_mag=None);

Selects earthquakes whose magnitudes are within the range specified by the lower limit (\verb=lower_mag=) and the upper limit (\verb=upper_mag=).

\section{Declustering}

To identify Poissonian rate of seismicity, it is necessary to remove foreshocks/aftershocks/swarms from the catalogue. The Modeller's Toolkit contains, at present, two algorithms to undertake this task, with more under development.

\subsection{\textcite{GardnerKnopoff1974}}

The most widely applied simple windowing algorithm is that of 
\textcite{GardnerKnopoff1974}. Originally conceived for Southern California, 
the method simply identifies aftershocks by virtue of fixed time-distance
windows proportional to the magnitude of the main shock. Whilst this 
premise is relatively simple, the manner in which the windows are 
applied can be ambiguous. Four different possibilities can be 
considered \parencite{LuenStark2012}:

\begin{enumerate}
\item Search events in magnitude-descending order. Remove events if it is 
    in the window of the largest event
\item Remove every event that is inside the window of a previous event, 
    including larger events
\item An event is in a cluster if, and only if, it is in the window of at 
    least one other event in the cluster. In every cluster remove all 
    events except the largest
\item In chronological order, if the $i^{th}$ event is in the window of a 
    preceding larger shock that has not already been deleted, remove it. 
    If a larger shock is in the window of the $i^{th}$ event, delete the 
    $i^{th}$ event. Otherwise retain the $i^{th}$ event.
\end{enumerate}

It is the first of the four options that is implemented in the current 
toolkit, whilst others may be considered in future.  The algorithm is 
capable if identifying foreshocks and aftershocks, simply by applying 
the windows forward and backward in time from the mainshock. 
No distinction is made between primary aftershocks (those resulting 
from the mainshock) and secondary or tertiary aftershocks (those 
originating due to the previous aftershocks); however, it is assumed 
all would occur within the window.

Several modifications to the time and distance windows have been 
suggested, which are summarised in \textcite{vanStiphout2012}. The windows 
originally suggested by \textcite{GardnerKnopoff1974} are approximated by:

\begin{equation}\begin{split} 
\mbox{distance (km)} = &10^{0.1238 M + 0.983}\\
\mbox{time (decimal years)} = & 
\begin{cases} 10^{0.032 M + 2.7389} & \text{if $M \geq 6.5$} \\ 
              10^{0.5409 M - 0.547} & \mbox{otherwise}  \end{cases}\end{split}
\end{equation}

An alternative formulation is proposed by Gr\"unthal (as reported in \textcite{vanStiphout2012}):

\begin{equation}\begin{split} 
\mbox{distance (km)} = & e^{1.77 + \left( {0.037 + 1.02 M} \right)^2} \\ 
   \mbox{time (decimal years)} = & \begin{cases}   |e^{-3.95+ \left( {0.62 + 17.32 M}
    \right)^2}|    & \text{if $M \geq 6.5$ } \\ 10^{2.8 + 0.024 M} & 
    \text{otherwise}  \end{cases}\end{split}
\end{equation}
A further alternative is suggested by \textcite{Uhrhammer1986}
%
\begin{equation}
\mbox{distance (km)} = e^{-1.024 + 0.804 M} \quad \mbox{time (decimal years)} = 
    e^{-2.87 + 1.235 M}
\end{equation}

A comparison of the expected window sizes with magnitude are shown for 
distance  and time (Figure \ref{fig:declust_scaling}).

\begin{figure}[htb]
  \centering
  \begin{subcaption}
      \centering
      \includegraphics[width=8cm]{./figures/declustering_distance_windows.eps}
	\end{subcaption}
  \begin{subcaption}
      \centering
      \includegraphics[width=8cm]{./figures/declustering_time_windows.eps}
	\end{subcaption}	
	\caption{Scaling of declustering time and distance windows with magnitude}
	\label{fig:declust_scaling}
\end{figure}

The \textcite{GardnerKnopoff1974} algorithm and its derivatives represent 
are most computationally straightforward approach to declustering. The \verb=time_dist_windows= attribute indicates the choice of the 
time and distance window scaling model from the three listed. As 
the current version of this algorithm considers the events in a 
descending-magnitude order, the parameter \verb=foreshock_time_window= 
defines the size of the time window used for searching for foreshocks, 
as a fractional proportion of the size of the aftershock window (the 
distance windows are always equal for both fore- and aftershocks). 
So for an evenly sized time window for foreshocks and aftershocks,the\\
\verb=foreshock_time_window= parameter should equal 1. For shorter or longer 
foreshock time windows this parameter can be reduced or increased respectively.

To run a declustering analysis on the earthquake catalogue it is necessary to set-up the configuration using a python dictionary (see Appendix \ref{sec:python_guide}). A config file for the \textcite{GardnerKnopoff1974} algorithm, using for example the \textcite{Uhrhammer1986} time-distance windows with equal sized time window for aftershocks and foreshocks, would be created as shown:

\begin{python}[frame=single]
>> from openquake.hmtk.seismicity.declusterer.distance_time_windows import\
    UhrhammerWindow

>> declust_config = {
    'time_distance_window': UhrhammerWindow(),
    'fs_time_prop': 1.0}
\end{python}


To run the declustering algorithm simply import and run the algorithm as shown:

\begin{python}[frame=single]
>> from openquake.hmtk.seismicity.declusterer.dec_gardner_knopoff import\
    GardnerKnopoffType1

>> declustering = GardnerKnopoffType1()

>> cluster_index, cluster_flag = declustering.decluster(
    catalogue,
    declust_config)
\end{python}

There are two outputs of a declustering algorithm: \verb=cluster_index= and \verb=cluster_flag=. Both are numpy vectors, of the same length as the catalogue, containing information about the clusters in the catalogue. \verb=cluster_index= indicates the cluster to which each event is assigned (0 if not assigned to a cluster). \verb=cluster_flag= indicates whether an event is a non-Poissonian event, in which case the value is assigned to 1, or a mainshock, the value is assigned as 0. This output definition is the same for all declustering algorithms.

At this point the user may wish to either retain the catalogue in its current format, in which case they may wish to add on the clustering information into another attribute of the catalogue.data dictionary, or they may wish to purge the catalogue of non-Poissonian events. 

To simply add the clustering information to the data dictionary simply type:

\begin{python}[frame=single]
>> catalogue.data['Cluster_Index'] = cluster_index
>> catalogue.data['Cluster_Flag'] = cluster_flag
\end{python}
 
Alternatively, to purge the catalogue of non-Poissonian events:

\begin{python}[frame=single]
>> mainshock_flag = cluster_flag == 0
>> catalogue.purge_catalogue(mainshock_flag)
\end{python}


\subsection{AFTERAN \parencite{Musson1999PSHABalkan}}

A particular development of the standard windowing approach is introduced in the program AFTERAN \parencite{Musson1999PSHABalkan}. This is a modification of the \textcite{GardnerKnopoff1974} algorithm, using a moving time window rather than a fixed time window. In AFTERAN, considering each earthquake in order of descending magnitude, events within a fixed distance window are identified (the distance window being those suggested previously). These events are searched using a moving time window of T days. For a given mainshock, non Poissonian events are identified if they occur both within the distance window and the initial time window. The time window is then moved, beginning at the last flagged event, and the process repeated. For a given mainshock, all non-Poissonian events are identified when the algorithm finds a continuous period of T days in which no aftershock or foreshock is identified. 

The theory of the AFTERAN algorithm is broadly consistent with that of \textcite{GardnerKnopoff1974}. This algorithm, whilst a little more computationally complex, and therefore slower, than the \textcite{GardnerKnopoff1974} windowing approach, remains simple to implement. 

As with the \textcite{GardnerKnopoff1974} function, the \verb=time_dist_window= attribute indicates the choice of the time and distance window scaling model. The parameter \verb=time_window= indicates the size (in days) of the moving time window used to identify fore- and aftershocks. The following example will show how to run the AFTERAN algorithm, using the  \textcite{GardnerKnopoff1974} definition of the distance windows, and a fixed-width moving time window of 100 days.

  
\begin{python}[frame=single]

>> from openquake.hmtk.seismicity.declusterer.dec_afteran import\
    Afteran

>> from openquake.hmtk.seismicity.declusterer.distance_time_windows import\
    GardnerKnopoffWindow   
 
>> declust_config = {
    'time_distance_window': GardnerKnopoffWindow(),
    'time_window': 100.0} 

>> declustering = Afteran()

>> cluster_index, cluster_flag = declustering.decluster(
    catalogue,
    declust_config)
\end{python} 

%::::::::::::::::::::::::::::::::::::::::::::::::::::::::::::::::::::::::::::::::::::::::::::::::::::::::::::::::::::::::::::::::::::::::::::::::::::::::::::::

\section{Completeness}

In the earliest stages of processing an instrumental seismic catalogue to derive inputs for seismic hazard analysis, it is necessary to determine the magnitude completeness threshold of the catalogue. To outline the meaning of the term ''magnitude completeness'' and the requirements for its analysis as an input to PSHA, the terminology of \textcite{MignanWoessner2012} is adopted. This defines the magnitude of completeness as the ''lowest magnitude at which 100 \% of the events in a space-time volume are detected \parencite{RydelekSacks1989, WoessnerWiemer2005}''. Incompleteness of an earthquake catalogue will produce bias when determining models of earthquake recurrence, which may have a significant impact on the estimation of hazard at a site. Identification of the completeness magnitude of an earthquake catalogue is therefore a clear requirement for the processing of input data for seismic hazard analysis.

It should be noted that this summary of methodologies for estimating completeness is directed toward techniques that can be applied to a ''typical'' instrumental seismic catalogue. We therefore make the assumption that the input data will contain basic information for each earthquake such as time, location, magnitude. We do not make the assumption that network-specific or station-specific properties (e.g., configuration, phase picks, attenuation factors) are known a priori. This limits the selection of methodologies to those classed as estimators of ''sample completeness'', which defines completeness on the basis of the statistical properties of the earthquake catalogue, rather than ''probability-based completeness'', which defines the probability of detection given knowledge of the properties of the seismic network \parencite{SchorlemmerWoessner2008}. This therefore excludes the methodology of \textcite{SchorlemmerWoessner2008}, and similar approaches such as that of \textcite{Felzer2008}

The current workflows assume that completeness will be applied to the whole catalogue, ideally returning a table of time-varying completeness. The option to explore spatial variation in completeness is not explicitly supported, but could be accommodated by an appropriate configuration of the toolkit.

In the current version of the Modeller's Toolkit the \textcite{Stepp1971} methodology for analysis of catalogue completeness is implemented. Further methods are in development, and will be input in future releases.

%\subsection{User-defined Table}
%
%This is simply a filtering that will remove from further consideration any events outside of the completeness bounds defined by the user. The table represents the time variation in $M_C$ and can be input as a separate file (in comma-separated value format) in the following format.
%\begin{Verbatim}[frame=single, commandchars=\\\{\}, fontsize=\scriptsize]
%1990.0, 4.0\\
%1960.0, 5.0\\
%1900.0, 6.0\\
%1700.0, 7.0\\
%\end{Verbatim}
%
%The left-hand column represents the earliest year at which the earthquake is complete at the corresponding magnitude in the right-hand column. \\
%\textbf{Important: The values in the completeness file must be entered from most-recent to oldest!} 

\subsection{\cite{Stepp1971}}

This is one of the earliest analytical approaches to estimation of completeness magnitude. It is based on estimators of the mean rate of recurrence of earthquakes within given magnitude and time ranges, identifying the completeness magnitude when the observed rate of earthquakes above $M_C$ begins to deviate from the expected rate. If a time interval ($T_i$) is taken, and the earthquake sequence assumed Poissonian, then the unbiased estimate of the mean rate of events per unit time interval of a given sample is:

\begin{equation}
   \lambda = \frac{1}{n} \sum_{i = 1}^{n} T_i
\end{equation}

with variance $\sigma_{\lambda}^{2} = \lambda / n$. Taking the unit time interval to be 1 year, the standard deviation of the estimate of the mean is:

\begin{equation}
   \sigma_{\lambda} = \sqrt{\lambda} / \sqrt{T}
\end{equation}

where $T$ is the sample length. As the Poisson assumption implies a stationary process, $\sigma_{\lambda}$ behaves as $1/\sqrt{T}$ in the sub-interval of the sample in which the mean rate of occurrence of a magnitude class is constant. Time variation of $M_C$ can usually be inferred graphically from the analysis, as is illustrated in Figure \ref{fig:SteppFigExample1}. In this example, the deviation from the $1/\sqrt{T}$ line for each magnitude class occurs at around 40 years  for $4.5 < M < 5$, 100 years for $5.0  < M < 6.0$, approximately 150 years for $6.0 < M < 6.5$ and 300 years for $M > 6.5$. Knowledge of the sources of earthquake information for a given catalogue may usually be reconciled with the completeness time intervals.

\begin{figure}[htb]
	\centering
		\includegraphics[height=10cm, keepaspectratio=true]{./figures/C2Fig1SteppFig1.eps}
	\caption{Example of Completeness Estimation by the \textcite{Stepp1971} methodology}
	\label{fig:SteppFigExample1}
\end{figure}

The analysis of \textcite{Stepp1971} is a coarse, but relatively robust, approach to estimating the temporal variation in completeness of a catalogue. It has been widely applied since its development. The accuracy of the completeness magnitude depends on the magnitude and time intervals considered, and a degree of judgement is often needed to determine the time at which the rate deviates from the expected values. It has tended to be applied to catalogues on a large scale, and for relatively higher completeness magnitudes. 

To translate the methodology from a largely graphical methods into a computational method the completeness period needs to be identified by automatically identifying the point at which the gradient of the observed values decreases with respect to that expected from a Poisson process (see \ref{fig:SteppFigExample1}). In the implementation found within the current toolkit, the divergence point is identified by fitting a two-segment piecewise linear function to the observed data. Although a two-segment piecewise linear function is normally fit with four parameters (intercept, $slope_1$, $slope_2$ and crossover point), by virtue of the assumption that for the complete catalogue the rate is assumed to be stationary such that $\sigma_{\lambda} = \frac{1}{\sqrt{T}}$ the slope of the first segment can be fixed as $-0.5$, and the second slope should be constrained such that $slope_2 \leq -0.5$, whilst the crossover point ($x_c$) is subject to the constraint ($x_c \geq 0.0$). Thus it is possible to fit the two-segment linear function using constrained optimisation with only three free parameters. For this purpose the toolkit minimises the residual sum-of-squares of the model fit using numerical optimisation. 

To run the \textcite{Stepp1971} algorithm the configuration parameters should be entered in the form of a dictionary, such as the example shown below:

\begin{python}[frame=single]
comp_config = {'magnitude_bin': 0.5,
               'time_bin': 5.,
               'increment_lock': True}
\end{python}

The algorithm has three configurable options. The \verb=time_bin= parameter describes the size of the time window in years, the \verb=magnitude_bin= parameter describes the size of the magnitude bin, sensitivity is as described previously. The final option (\verb=increment_lock=) is an option that is used to ensure consistency in the results to avoid the completeness magnitude increasing for the latest intervals in the catalogue simply due to the variability associated with the short duration. If \verb=increment_lock= is set to \verb=True=, the program will ensure that the completeness magnitude for shorter, more recent windows is less than or equal to that of older, longer windows. This is often a condition for some recurrence analysis tools, so it may be advisable to set this option to true in certain workflows. Otherwise it should be set to \verb=False to show the apparent variability=. Some degree of judgement is necessary here. In particular it is expected that the user may be aware of circumstances particular to their catalogue for which a recent increase in completeness magnitude is expected (for example, a certain recording network no longer operational).  


The process of running the algorithm is shown below:

\begin{python}[frame=single]
>> from openquake.hmtk.seismicity.completeness.comp_stepp_1971 import\
    Stepp1971

>> completeness_algorithm = Stepp1971()

>> completeness_table = completeness_algorithm.completeness(
    catalogue,
    comp_config)

>> completeness_table 
array([[ 1990.  ,     4.25],
       [ 1962.  ,     4.75],
       [ 1959.  ,     5.25],
       [ 1906.  ,     5.75],
       [ 1906.  ,     6.25],
       [ 1904.  ,     6.75],
       [ 1904.  ,     7.25]])
\end{python}

As shown in the resulting \verb=completeness_table=, the completeness algorithm will output the time variation in completeness (in this example with the \verb=increment_lock= set) in the form of a two-column table with column 1 indicating the completeness year for the magnitude bin centred on the magnitude value found in column 2.

At present, it may be the case that the user wishes to enter a time-varying completeness results for use in subsequent functions, based on alternative methods or on judgement. This can be entered in the \verb=completeness_table= setting, as in the example shown here (take note of the requirements for the square brackets):

\begin{python}[frame=single]
completeness_table: [[1990., 4.0],
                     [1960., 5.0],
                     [1930., 6.0],
                     [1900., 6.5]]
\end{python}

If a \verb=completeness_table= is input then this will override the selection of the completeness algorithm, and the calculation will take the values in \verb=completeness_table= directly. 

%::::::::::::::::::::::::::::::::::::::::::::::::::::::::::::::::::::::::::::::::::::::::::::::::::::::::::::::::::::::::::::::::::::::::::::::::::::::::::::::

\section{Recurrence Models}

The current sets of tools are intended to determine the parameters of the \textcite{GutenbergRichter1944} recurrence model, namely the a- and b-value. It is expected that in the most common use case the catalogue that is input to these algorithms will be declustered, with a time-varying completeness defined according to a \verb=completeness_table= of the kind shown previously. If no \verb=completeness_table= is input the algorithm will assume the input catalogue is complete above the minimum magnitude for its full duration.

\subsection{\textcite{Aki1965}}

The classical maximum likelihood estimator for a simple unbounded \textcite{GutenbergRichter1944} model is that of \textcite{Aki1965}, adapted for binned magnitude data by \textcite{Bender1983}. It assumes a fixed completeness magnitude ($M_C$) for the catalogue, and a simple power law recurrence model. It does not explicitly take into account magnitude uncertainty.

\begin{equation}
   b = \frac{ \log_{10} \left( e \right)}{ \bar{m} - m_0 + \left( {\frac{\Delta M}{2}} \right)}
\end{equation}

\noindent where $\bar{m}$ is the mean magnitude, $m_0$ the minimum magnitude and $\Delta M$ the discretisation interval of magnitude within a given sample.

\subsection{Maximum Likelihood}

This method adjusts the \textcite{Aki1965} and \textcite{Bender1983} method to incorporate for time variation in completeness. The catalogue is divided into
into S sub-catalogues, where each sub-catalogue corresponds to a period 
with a corresponding $M_C$.  An average a- and b-value (with uncertainty) is returned by taking 
the mean of the a- and b-value of each sub-catalogue, weighted by 
the number of events in each sub-catalogue.

\begin{equation}
   \hat{b} = \frac{1}{S} \sum_{i = 1}^{S} w_i b_i
\end{equation}

\begin{python}[frame=single]
>> mle_config = {'magnitude_interval': 0.1,
                 'Average Type': 'Weighted',
                 'reference_magnitude': None}

>> from openquake.hmtk.seismicity.occurrence.b_maximum_likelihood import\
    BMaxLikelihood

>> recurrence = BMaxLikelihood()

>> bval, sigmab, aval, sigmaa = recurrence.calculate(
    catalogue,
    mle_config, 
    completeness=completeness_table)
\end{python}

Where \verb=magnitude_window= indicates the size of the magnitude bin, \verb=recurrence_algorithm= and \verb=reference_magnitude= the magnitude for which the output calculates that rate greater than or equal to (set to \verb=0= for $10^{a}$). 

\subsection{\textcite{KijkoSmit2012}}

A recent adaption of the \textcite{Aki1965} estimator of b-value for  a catalogue containing different completeness periods has been proposed by \textcite{KijkoSmit2012}. Dividing the earthquake catalogue into $s$ subcatalogues of $n_i$ events with corresponding completeness magnitudes $m_{c_i}$ for $i = 1, 2, ..., s$, the likelihood function of $\beta$ where $\beta = b \ln		 \left( {10.0} \right)$ is given as:

\begin{equation}
    \mathbf{L} = \prod_{i = 1}^{s} \prod_{j = 1}^{n_i} \beta \exp(\left[ {-\beta \left( {m_j^i - m_{min}^i } \right) } \right])
\end{equation}

\noindent which gives a maximum likelihood estimator of $\beta$:

\begin{equation}
    \beta = \left( {\frac{r_1}{\beta_1} + \frac{r_2}{\beta_2} + \dots + \frac{r_s}{\beta_s}} \right)^{-1}
\end{equation}

\noindent where $r_i = n_i / n$ and $n = \sum_{i = 1}^{s} n_i$ above the level of completeness $m_i$.

\begin{python}[frame=single]]

>> kijko_smit_config = {'magnitude_interval': 0.1,
                        'reference_magnitude': None\}
\end{python}

\subsection{\textcite{Weichert1980}}

Recognising the typical conditions of an earthquake catalogue, \textcite{Weichert1980} developed a maximum likelihood estimator of $b$ for grouped magnitudes and unequal periods of observation. The likelihood formulation for this approach is:

\begin{equation}
   \mathbf{L} \left( {\beta | n_i, m_i, t_i} \right) = \frac{ N!}{\prod_i n_i!} \prod_i p_{i}^{n_i}
\end{equation}

where $\mathbf{L}$ is the likelihood estimator of $\beta$, $n$ the number of earthquakes in magnitude bin m with observation period t. The parameter $p$ is defined as:

\begin{equation}
   p_i = \frac{t_i \exp \left( {-\beta m_i} \right) }{\sum_j t_j \exp \left( {-\beta m_j} \right)}
\end{equation}

The extremum of $\ln \left( {\mathbf{L}}\right)$ is found at:

\begin{equation} 
   \frac{\sum_i t_i m_i \exp \left( {-\beta m_i} \right)}{\sum_j t_j \exp \left( {-\beta m_j} \right)}
\end{equation}

The computational implementation of this method is given as an appendix to \textcite{Weichert1980}. This formulation of the maximum likelihood estimator for b-value, and consequently seismicity rate, is in widespread use, with applications in many national seismic hazard analysis \parencite[e.g.][]{usgsNSHM1996,usgsNSHM2002}. The algorithm has been demonstrated to be efficient and unbiased for most applications. It is recognised by \textcite{Felzer2008} that an implicit assumption is made regarding the stationarity of the seismicity for all the time periods. 

To implement the \textcite{Weichert1980} recurrence estimator, the configuration properties are defined as:

\begin{python}[frame=single]

>> weichert_config = {`magnitude_interval': 0.1,
                      `reference_magnitude': None,
                      # The remaining parameters are optional
                      `bvalue': 1.0,
                      `itstab': 1E-5,
                      `maxiter': 1000}
\end{python}

As the \textcite{Weichert1980} algorithm is reaches the MLE estimation by iteration then three additional optional parameters can control the iteration process: \verb=bvalue= is the initial guess for the b-value, \verb=itstab= the difference in b-value in order to reach convergence, and \verb=maxiter= the maximum number of iterations. \footnote{The iterative nature of the \textcite{Weichert1980} algorithm can result in very slow convergence and unstable behaviour when the magnitudes infer b-values that are very small, or even negative. This can occur when very few events are in the resulting catalogue, or when the magnitudes converge within a narrow range.}



%::::::::::::::::::::::::::::::::::::::::::::::::::::::::::::::::::::::::::::::::::::::::::::::::::::::::::::::::::::::::::::::::::::::::::::::::::::::::::::::
\section{Maximum Magnitude}

The estimation of the maximum magnitude for use in seismic hazard analysis is a complex, and often controversial, process that should be guided by information from geology and the seismotectonics of a seismic source. Estimation of maximim magnitude from the observed (instrumental and historical) seismicity can be undertaken using methods assuming a truncated \cite{GutenbergRichter1944} model, or via non-parametric methods that are independent any assumed functional form. 

\subsection{\textcite{Kijko2004}}

Three different estimators of maximum magnitude are given by \textcite{Kijko2004}, each depending on a different set of assumptions:
\begin{enumerate}
\item ''Fixed b-value'': Assumes a single b-value with no uncertainty 
\item ''Uncertain b-value'': Assumes and uncertain b-value defined by an expected b and the standard deviation
\item ''Non-Parametric Gaussian'': Assumes no functional form (can be applied to seismicity observed to follow a more characteristic distribution)
\end{enumerate}

Each of these estimators assumes the general form:

\begin{equation}
m_{max} = m_{max}^{obs} + \Delta
\end{equation}

where $\Delta$ is an increment that is dependent on the estimator used.

The uncertainty on $m_{max}$ is also defined according to:

\begin{equation}
    \sigma_{m_{max}} = \sqrt{\sigma_{m_{max}^{obs}}^2 + \Delta^{2}}
\end{equation}

In the three estimators some lower bound magnitude constraint must be defined. For those estimators that assume an exponential recurrence model the lower bound magnitude must be specified by the users. For the non-Parametric Gaussian method and explicit lower bound magnitude does not have to be specified; however, the estimation is conditioned upon the largest N magnitudes, where N must be specified by the user.

If the user wishes to input a maximum magnitude that is larger than that observed in the catalogue (e.g. a known historical magnitude), this can be specified in the config file using \verb=input_mmax= with the corresponding uncertainty defined by \\ \verb=input_mmax_uncertainty=. If these are not defined (i.e. set to \verb=None=) then the maximum magnitude will be taken from the catalogue.

All three estimators require an iterative solution, therefore additional parameters can be specified in the configuration file that control the iteration process: \verb=tolerance= difference in $M_Max$ estimate for the algorithm to be considered converged, and \\ \verb=maximum_iterations= the maximum number of iterations for stability. 


\subsubsection{''Fixed b-value''}

For a catalogue of $n$ earthquakes, whose magnitudes are distributed by a \textcite{GutenbergRichter1944} distribiution with a fixed "b" value, the increment of maximum magnitude is determined via:

\begin{equation}
\Delta = \int\limits_{m_{min}} ^{m_{max}} \left[ {\frac{1 - \exp \left[ {-\beta \left( {m - m_{min}} \right)} \right]}{1 - \exp \left[ {-\beta \left( {m_{max}^{obs} - m_{min}} \right) } \right]}} \right] ^n dm
\end{equation}


The execution of the \textcite{Kijko2004} ''fixed-b'' algorithm is as follows:

\begin{python}[frame=single]
>> mmax_config = {`input_mmax': 7.6,
                  `input_mmax_uncertainty': 0.22,
                  `b-value': 1.0,
                  `input_mmin': 5.0,
                  `tolerance': 1.0E-5,  \# Default
                  `maximum_iterations': 1000\} \# Defaults
                       
>> from openquake.hmtk.seismicity.max_magnitude.kijko_sellevol_fixed_b\
    import KijkoSellevolFixedb

>> mmax_estimator = KijkoSellevolFixedb()

>> mmax, mmax_uncertainty = mmax_estimator.get_mmax(catalogue,
                                                    mmax_config)   
\end{python}

\subsubsection{''Uncertain b-value''}

For a catalogue of $n$ earthquakes, whose magnitudes are distributed by a \textcite{GutenbergRichter1944} distribiution with an uncertain "b" value, characterised by and expected term ($b$) and a corresponding undertainty ($\sigma_b$), the increment of maximum magnitude is determined via:


\begin{equation}
\Delta = \left( {C_{\beta}} \right)^n \int\limits_{m_min}^{m_max} \left[ {1 - \left( {\frac{p}{p + m - m_{min}}} \right) ^q} \right]^n dm
\end{equation}

where $\beta = b \ln \left( {10.0} \right)$, $p = \beta / \left( {\sigma_{\beta}} \right) ^ 2$, $q = \left( {\beta / \sigma_{\beta}} \right) ^ 2$ and $C_{\beta}$ is a normalising coefficient determined via:

\begin{equation}
C_{\beta} = \frac{1}{1 - \left[ {p / \left( {p + m_{max} - m_{min}} \right) } \right]^q}
\end{equation}

In both the fixed and uncertain ''b'' case a minimum magnitude will need to be input into the calculation. If this value is lower than the minimum magnitude observed in the catalogue the iterator may not stabilise to a satisfactory value, so it is recommended to use a minimum magnitude that is greater than the minimum found in the observed catalogue.

The execution of the ''uncertain b-value'' estimator is undertaken in a very similar to that of the fixed b-value, the only additional parameter being the \verb=sigma-b= term:

\begin{python}[frame=single]
>> mmax_config = {'input_mmax': 7.6,
                  'input_mmax_uncertainty': 0.22,
                  'b-value': 1.0,
                  'sigma-b': 0.15
                  'input_mmin': 5.0,
                  'tolerance': 1.0E-5, 
                  'maximum_iterations': 1000}
                       
>> from openquake.hmtk.seismicity.max-magnitude.kijko_sellevol_bayes\
    import KijkoSellevolBayes

>> mmax_estimator = KijkoSellevolBayes()

>> mmax, mmax_uncertainty = mmax_estimator.get_mmax(catalogue,
                                                    mmax_config)
\end{python}

\subsubsection{Non-Parametric Gaussian}

The non-parametric Gaussian estimator for maximum magnitude $m_{max}$ is defined as:

\begin{equation}
\Delta = \int\limits_{m_{min}}^{m_{max}} \left[ {\frac{\sum_{i = 1}^{n} \left[ {\Phi \left( {\frac{m - m_i}{h}} \right) - \Phi \left( {\frac{m_{min} - m_i}{h}} \right)} \right]}{\sum_{i = 1}^{n} \left[ {\Phi \left( {\frac{m_{max} - m_i}{h}} \right) - \Phi \left( {\frac{m_{min} - m_i}{h}} \right)} \right]}} \right]^n  dm
\end{equation}
where $m_{min}$ and $m_{max}$ are the minimum and maximum magnitudes from a set of $n$ events, $\Phi$ is the standard normal cumulative distribution function. $h$ a kernel smoothing factor:
\begin{equation}
h = 0.9 \times min\left( {\sigma, IQR / 1.34} \right) \times n^{-1 / 5}
\end{equation}
with $\sigma$ the standard deviation of a set of n earthquakes with magnitude $m_{i}$ where $i = 1, 2, ... n$, and $IQR$ the inter-quartile range. 

Therefore the uncertainty on $m_{max}$ is conditioned primarily on the uncertainty of the largest observed magnitude. As in many catalogues the largest observed magnitude may be an earlier historical event, which will be associated with a large uncertainty, this estimator tends towards large uncertainties on $m_{max}$.

Due to the need to define some additional parameters the configuration file is slightly different. No b-value or minimum magnitude needs to be specified; however, the algorithm will consider only the largest \verb=number_earthquakes= magnitudes (or all magnitudes if the number of observations is smaller). The algorithm also numerically approximates the integral of the Gaussian pdf, so \verb=number_samples= is the number of samples of the distribution. The rest of the execution remains the same as for the exponential recurrence estimators of $M_{max}$:

\begin{python}[frame=single]
>> mmax_config = {'input_mmax': 7.6,
                  'input_mmax_uncertainty': 0.22,
                  'number_samples': 51, # Default
                  'number_earthquakes': 100 # Default 
                  'tolerance': 1.0E-5, 
                  'maximum_iterations': 1000}
                       
>> from openquake.hmtk.seismicity.max-magnitude.kijko_nonparametric_gaussian\
     import KijkoNonParametricGaussian

>> mmax_estimator = KijkoNonParametricGaussian()

>> mmax, mmax_uncertainty = mmax_estimator.get_mmax(catalogue,
                                                    mmax_config)
                
\end{python}


\subsection{Cumulative Moment \parencite{MakropoulosBurton1983}}

The cumulative moment release method is an adaptation of the cumulative strain energy release method for estimating $m_{max}$ originally proposed by \textcite{MakropoulosBurton1983}. Another method based on a pseudo-graphical formulation, an estimator of maximum magnitude can be derived from a plot of cumulative seismic moment release with time. The average slope of this plot indicates the mean moment release for the input catalogue in question. Two further straight lines are defined with gradients equal to that of the slope of mean cumulative moment release, both enveloping the cumulative plot. The vertical distance between these two lines indicates the total amount of moment that may be released in the region, if no earthquakes were to occur in the corresponding time (i.e. the distance between the upper and lower bounding lines on the time axis). This concept is illustrated in Figure \ref{fig:Cumulative_Moment}. 

\begin{figure}[htb]
	\centering
		\includegraphics[height=6cm, keepaspectratio=true]{./figures/Cumulative_Moment.eps}
	\caption{Illustratation of Cumulative Moment Release Concept}
	\label{fig:Cumulative_Moment}
\end{figure}

The cumulative moment estimator of $m_{max}$, whilst simple in concept, has several key advantages. As a non-parametric method it is independent of any assumed probability distribution and cannot estimate $m_{max}$ lower than the observed $m_{max}$. It is also principally controlled by the largest events in the catalogue, this making it relative insensitive to uncertainties in completeness or lower bound threshold. In practice, this estimator, and to some extent that of \textcite{Kijko2004} are dependent on having a sufficiently long record of events relative to the strain cycle for the region in question, such that the estimate of average moment release is stable. This will obviously depend on the length of the catalogue, and for some regions, particularly those in low strain intraplate environments, it is often the case that $m_{max}$ will be close to the observed $m_{max}$. Therefore it may be the case that it is most appropriate to use these techniques on a larger scale, either considering multiple sources or an appropriate tectonic proxy.

For the cumulative moment estimator it is possible to take into account the uncertainty on $m_{max}$ by applying bootstrap sampling to the observed magnitudes and their respective uncertainties. This has the advantage that $\sigma_{m_{max}}$ is not controlled by the uncertainty on the observed $m_{max}$, as it is for the \textcite{Kijko2004} algorithm. Instead it takes into account the uncertainty on all the magnitudes in the catalogue. The cost of this, however, is that this method is more computationally intensive, and therefore slower, than \textcite{Kijko2004}, depending on the number of bootstrap samples the user chooses.

The algorithm is slightly simpler to run than the \textcite{Kijko2004} methods; however, due to the bootstrapping process it is slightly slower. It is run as per the following example:

\begin{python}[frame=single]

>> mmax_config = {'number_bootstraps': 1000}
                       
>> from openquake.hmtk.seismicity.max_magnitude.cumulative_moment_release\
     import CumulativeMoment

>> mmax_estimator = CumulativeMoment()

>> mmax, mmax_uncertainty = mmax_estimator.get_mmax(catalogue,
                                                    mmax_config)
                
\end{python}

For the cumulative moment algorithm the only user configurable parameter is the \\ \verb=number_bootstraps=, which is the number of samples used during the bootstrapping process. 

\section{Smoothed Seismicity}

The use of smoothed seismicity in seismic hazard analysis has generally become a common way of characterising distributed seismicity, for which the seismogenic source are defined exclusively from the uncertain locations of observed seismicity. There are many different methods for smoothing the catalogue, adopting different smoothing kernels or making different correction factors to compensate for spatial and/or temporal completeness. 

\subsection{\textcite{frankel1995}}

A smoothed seismicity method that has one of the clearest precedents for use in seismic hazard analysis is that of \textcite{frankel1995}, originally derived to characterise the seismicity of the Central and Eastern United States as part of the 1996 National Seismic Hazard Maps of the United States. The method applies a simple isotropic Gaussian smoothing kernel to derive the expected rate of events at each cell $\tilde{n}_i$ from the observed rate $n_j$ of seismicity in a grid of $j$ cells. This kernel takes the form:

\begin{equation}
\tilde{n_i} = \frac{\sum_j n_j e^{d_{ij}^2 / c^2}}{\sum_j e^{d_{ij}^2 / c^2}} 
\end{equation}

In the implementation of the algorithm, two steps are taken that we prefer to make configurable options here. The first step is that the time-varying completeness is accounted for using a correction factor ($t_f$) based on the \textcite{Weichert1980} method:

 \begin{equation}
 t_f = \frac{\sum_i e^{-\beta m_{c_i}}}{\sum_i T_i e^{-\beta m_{c_i}}} 
 \end{equation}
 
where $m_{c_i}$ the completeness magnitude corresponding to the mid-point of each completeness interval, and $T_i$ the duration of the completeness interval. The completeness magnitude bins must be evenly-spaced; hence, within the application of the progress a function is executed to render the input completeness table to one in which the magnitudes are evenly spaced with a width of 0.1 magnitude units. 

\subsection{Implementing the Smoothed Seismicity Analysis}

The smoothed seismicity separates out the core implementation (i.e. the gridding, counting and execution of the code) and the choice of kernel. An example of the execution process is as follows:

The first stage is to upload the catalogue into an instance of the catalogue class

\begin{python}[frame=single]
>> input_file = 'path/to/input_file.csv'

>> from openquake.hmtk.parsers.catalogue.csv_catalogue_parser import\
    CsvCatalogueParser

>> parser = CsvCatalogueParser(input_file)

>> catalogue = parser.read_file()
\end{python}

Next setup the smoothing algorithm using and the corresponding kernel:

\begin{python}[frame=single]

# Imports the smoothed seismicity algorithm
>> from openquake.hmtk.seismicity.smoothing.smoothed_seismicity import\
    SmoothedSeismicity

# Imports the Kernel function
>> from openquake.hmtk.seismicity.smoothing.kernels.isotropic_gaussian\
    import IsotropicGaussian

# Grid limits should be set up as 
# [min_long, max_long, spc_long, 
#  min_lat max_lat, spc_lat,
#  min_depth, max_depth, spc_depth]
>> grid_limits = [0., 10., 0.1, 0., 10., 0.1, 0., 60., 30.]
# Assuming a b-value of 1.0
>> smooth_seis = SmoothedSeismicity(grid_limits,
                                    use_3d=True,
                                    bvalue=1.0)
\end{python}

The smoothed seismicity function needs to be set up with three variables: i) the extent (and spacing) of the grid, ii) the choice to use 3D smoothing (i.e. distances are taken as hypocentral rather than epicentral) and iii) the input b-value. The extent of the grid can also be defined from the catalogue. If preferred the user need only specify the spacing of the longitude-latitude grid (as a single floating point value), then the grid will be defined by taking the bounding box of the earthquake catalogue and extended by the total smoothing length (i.e. the bandwidth (in km) multiplied by the maximum number of bandwidths). 

To run the smoothed seismicity analysis, the configurable parameters are: \verb=BandWidth= the bandwidth of the Gaussian kernel (in km), \verb=Length_Limit= the number of bandwidths considered as a maximum smoothing length, and \verb=increment= chooses whether to output the incremental a-value (for consistency with the original \textcite{frankel1995} methodology) or the cumulative a-value (corresponding to the a-value of the Gutenberg-Richter model).


The algorithm requires two essential inputs (the earthquake catalogue and the config file), and three optional inputs:

\begin{itemize}
\item \verb=completeness_table= A table of completeness magnitudes and their corresponding completeness years (as output from the completeness algorithms)

\item \verb=smoothing_kernel= An instance of the required smoothing
kernel class (currently only Isotropic Gaussian is supported - and will be used if not specified)

\item \verb=end_year= The final year of the catalogue. This will be taken as the last year found in the catalogue, if not specified by the user
\end{itemize}

The analysis is then run via:

\begin{python}[frame=single]
# Set up config (e.g. 50 km band width, up to 3 bandwidths)
>> config = {`Length_Limit': 3.,
             `BandWidth': 50.,
             'increment': True}
# Run the analysis!
>> output_data = smooth_seis.run_analysis(
    catalogue,
    config,
    completeness_table, 
    smoothing_kernel=IsotropicGaussian(), 
    end_year=None)

# To write the resulting data to a csv file
>> smooth_seis.write_to_csv(`path/to/output_file.csv')
\end{python}

The resulting output will be a csv file with the following columns:
\begin{Verbatim}[frame=single, commandchars=\\\{\}, fontsize=\scriptsize]
Longitude, Latitude, Depth, Observed Count, Smoothed Rate, b-value
\end{Verbatim}

\noindent where \verb=Observed Count= is the observed number of earthquakes in each cell, and \\ 
\verb=Smoothed Rate= is the smoothed seismicity rate.


%----------------------------------------------------------------------------------------
%	CHAPTER 3
%----------------------------------------------------------------------------------------
\chapterimage{./figures/chapter_head_1.pdf} % Chapter heading image
\chapter{Hazard Tools}
\label{chap:hazard}
\input{./hazard_tools.tex}

%----------------------------------------------------------------------------------------
%	CHAPTER 4
%----------------------------------------------------------------------------------------
\chapterimage{./figures/chapter_head_1.pdf} % Chapter heading image
\chapter{Geology Tools}
\label{chap:geology}
\input{./geology_tools.tex}

%----------------------------------------------------------------------------------------
%	CHAPTER 5
%----------------------------------------------------------------------------------------
\chapterimage{./figures/chapter_head_1.pdf} % Chapter heading image
\chapter{Geodetic Tools}
\label{chap:geodesy}
\input{./geodetic_tools.tex}

%----------------------------------------------------------------------------------------
%	CHAPTER 6
%----------------------------------------------------------------------------------------
%\chapterimage{./figures/chapter_head_1.pdf} % Chapter heading image
%\chapter{Hazard Applications}
%\label{chap:hazard}
%\input{./hazard.tex}


%----------------------------------------------------------------------------------------
%	BIBLIOGRAPHY
%----------------------------------------------------------------------------------------
\chapter*{Bibliography}
\addcontentsline{toc}{chapter}{\textcolor{ocre}{Bibliography}}
\section*{Books}
\addcontentsline{toc}{section}{Books}
\printbibliography[heading=bibempty,type=book]
\section*{Articles}
\addcontentsline{toc}{section}{Articles}
\printbibliography[heading=bibempty,type=article]
\section*{Other Sources}
\addcontentsline{toc}{section}{Reports}
\printbibliography[heading=bibempty,nottype=book,nottype=article]


%----------------------------------------------------------------------------------------
%	INDEX
%----------------------------------------------------------------------------------------

\cleardoublepage
\phantomsection
\setlength{\columnsep}{0.75cm}
\addcontentsline{toc}{chapter}{\textcolor{ocre}{Index}}
\printindex
\printglossary

%----------------------------------------------------------------------------------------


\part{Appendices}
\appendix
% -----------------------------------------------------------------------------
% -----------------------------------------------------------------------------
\chapter{The 10 Minute Guide to Python!}
\label{sec:python_guide}
%\begin{myfancybox}
% The objectives of this chapter are:
%\begin{itemize}
%     \item To introduce Python data types to facilitate use of the HMTK for Python beginners
% \end{itemize}
%\end{myfancybox}
\input{python_guide.tex}


\end{document}
















%\documentclass[11pt,a4paper,headings=small,dvips]{scrbook}
%\setcounter{secnumdepth}{3}
\setcounter{tocdepth}{3}
% This is used to create the cover and to plot trees
\usepackage{pst-tree}
\usepackage{pstricks,pstricks-add,multido}
%
\usepackage{geometry}
\usepackage{moresize}
%
%\usepackage{algorithmic}
% 
\usepackage{fancyvrb}
\usepackage{listings}
\usepackage{alltt}
\usepackage{gensymb}
%
\usepackage[section]{placeins}
% 
\usepackage{pbox}
% http://en.wikibooks.org/wiki/LaTeX/Indexing
\usepackage{makeidx} 
\makeindex
%
%\usepackage{subfigure}
% Figure caption settings
\usepackage[textfont=it,margin=10pt,font=small,labelfont=bf,labelsep=endash]{caption}
\usepackage{subcaption}
%
\usepackage{bm}
% Landscape package
\usepackage{lscape}
%
\usepackage{hyperref}
\hypersetup{
    colorlinks=true,
    linkcolor=blue,
    filecolor=magenta,
    urlcolor=cyan,
    breaklinks=true
}
%\hypersetup{colorlinks=true}
%\hypersetup{breaklinks=true}
% package for multiline comments
\usepackage{verbatim}
%
% Package to create a glossary - It must be uploaded after hyperref
% to produce the glossary: makeglossaries OQB
%\usepackage[toc,acronym,nonumberlist,style=altlist]{glossaries}
\usepackage[toc,nonumberlist,style=altlist]{glossaries}
\glstoctrue
\makeglossaries
%
% - - - - - - - - - - - - - - - - - - - - - - - - - - - - - - - - Setting Fonts
% \renewcommand{\encodingdefault}{OT1}
\renewcommand{\encodingdefault}{OT1}
\renewcommand{\familydefault}{ppl}
% \renewcommand{\familydefault}{cmss}
% \renewcommand{\seriesdefault}{m}
% \renewcommand{\shapedefault}{up}

% - - - - - - - - - - - - - - - - - - - - - - - - - - - - - - - - - - - - - - -
\usepackage{amsmath}
% - - - - - - - - - - - - - - - - - - - - - - - - - - - - - - - - - - - - - - -
\usepackage{titlesec}
\usepackage[dvips]{graphicx}
% - - - - - - - - - - - - - - - - - - - - - - - - - - - - - - - - - - - - - - -
\usepackage{type1cm,eso-pic,color}

%\makeatletter
%\AddToShipoutPicture{
%    \setlength{\@tempdimb}{.5\paperwidth}
%    \setlength{\@tempdimc}{.5\paperheight}
%    \setlength{\unitlength}{1pt}
%    \put(\strip@pt\@tempdimb,\strip@pt\@tempdimc){
%        \makebox(0,0){\rotatebox{55}{
%        	\textcolor[gray]{0.85}{
%        		\fontsize{5cm}{5cm}
%        		\selectfont{DRAFT}}
%        	}
%        }
%	}
%}
%\makeatother

%
% Solves problems with margin notes
\usepackage{mparhack} 
	\setlength{\marginparwidth}{1.1in}
	\let\oldmarginpar\marginpar
	\renewcommand\marginpar[1]{\-\oldmarginpar[\raggedright\color{red01}
	\footnotesize #1]%
	{\raggedright\footnotesize #1}}
% Define some colors
	\definecolor{azure}{RGB}{240,255,255}
	\definecolor{honeydew}{RGB}{240,255,240}
	\definecolor{blue01}{RGB}{4,64,116}
	\definecolor{blue02}{RGB}{0,62,113}
	\definecolor{gray01}{rgb}{0.1,0.1,0.1}
	\definecolor{gray02}{rgb}{0.8,0.8,0.8}
	\definecolor{red01}{rgb}{0.5,0.0,0.0}
	\definecolor{orange00}{rgb}{1.0,0.74,0.53}
	\definecolor{orange01}{rgb}{0.9137,0.5882,0.0980}
	\definecolor{orange02}{rgb}{0.7608,0.4157,0.1804}
	\definecolor{orange03}{rgb}{0.6941,0.1843,0.1333}
\usepackage[english]{babel}
% Bibliography settings
\usepackage[square,colon]{natbib} % Extend bibligraphy functions
% Page numbering by Chapter
%\usepackage[auto]{chappg} 
%\pagenumbering{bychapter}
% 
% Define page properties
\usepackage{scrpage2}
	\pagestyle{scrheadings}
	\lofoot[]{\includegraphics[width=2.0cm]{./figures/openquake_logo1.eps}}
	\refoot[]{\includegraphics[width=2.0cm]{./figures/openquake_logo1.eps}}
	%\renewcommand{\partpagestyle}{empty}
% - - - - - - - - - - - - - - - - - - - - - - - - - -  Reformatting PART Titles
\titleformat{\part}[display]
{\filleft\normalfont\sffamily}
{\textcolor{blue01}{\bfseries\large PART}\hspace{4pt}
	\bfseries\Huge\textcolor{blue01}{\thepart}}
{1pc}
{\Huge\bfseries\textcolor{blue01}}
[]
% - - - - - - - - - - - - - - - - - - - - - - - - - Reformatting CHAPTER Titles
% Titles: CHAPTER
\titleformat{\chapter}
	[display] % shape
	{\filleft\normalfont\sffamily} % format
	{\textcolor{blue01}{\bfseries\MakeUppercase{\chaptertitlename}} % label
	\hspace{4pt}\huge\bfseries\textcolor{blue01}{\thechapter}} 
	{1pc} % sep
	{\huge\bfseries\textcolor{blue01}} % Before
	[]
% - - - - - - - - - - - - - - - - - - - - - - - - - Reformatting SECTION Titles
% Titles: SECTION
\titleformat{\section}
	[hang] % shape
	{\vspace{.8ex}\Large\bfseries\color{blue01}} % format 
	{\textcolor{blue01}{\thesection.}} % label
	{.5em} % sep
	{} % before
	[] % after
% - - - - - - - - - - - - - - - - - - - - - - -  Reformatting SUBSECTION Titles
% Title: SUBSECTION
\titleformat{\subsection}
	[hang] % shape
	{\vspace{.8ex}\large\bfseries\color{blue01}} % format 
	{\textcolor{blue01}{\thesubsection.}} % label
	{.5em} % sep
	{} % before
	[] % after
%  - - - - - - - - - - - - - - - - - - - - -  Reformatting SUBSUBSECTION Titles 
% Title: SUBSUBSECTION
\titleformat{\subsubsection}
	[hang] % shape
	{\vspace{.8ex}\normalfont\bfseries\color{blue01}} % format 
	{\textcolor{blue01}{\thesubsubsection.}} % label
	{.5em} % sep
	{} % before
	[] % after
% - - - - - - - - - - - - - - - - - - - - - - -  Reformatting PARAGRAPH Titles 
% Title: PARAGRAPH
\titleformat{\paragraph}
	[hang] % shape
	{\vspace{.2ex}\normalfont\color{blue01}} % format 
	{} % label
	{} % sep
	{} % before
	[] % after
%

%\usepackage{xcolor}
%\usepackage{framed}
%\usepackage[utf8]{inputenc}
%\usepackage{listings}
%
%\newenvironment{myfancybox}{%
%  \def\FrameCommand{\fboxsep=\FrameSep \fcolorbox{blue01}{honeydew}}%
%  \color{black}\MakeFramed {\FrameRestore}}%
% {\endMakeFramed}
%
%\setlength{\parskip}{2.5mm}
%\setlength{\parindent}{0.0mm}
%
%\begin{document}
%\setcounter{page}{1}
%\lstset{language=Python}
%
%\begin{titlepage}
%	\title{ \textcolor{blue01}{\textsf{\bfseries\Huge 
%        Hazard Modeller's Toolkit
%        }}}
%	\subtitle{ \textcolor{blue01}{\textsf{\bfseries\LARGE
%        Documentation \& Tutorial}}}
%	\date{June 2014}
% 
%	\publishers{GEM Foundation, Pavia}
%\end{titlepage}
%\pagestyle{scrheadings}
%\maketitle
%% - - - - - - - - - - - - - - - - - - - - - - - - - - - - - -  Load the glossary
%%\input{glossary.tex}
%% -----------------------------------------------------------------------------
%% -----------------------------------------------------------------------------
%%\chapter*{Introduction}
%%\cleardoublepage
%% -----------------------------------------------------------------------------
%% -----------------------------------------------------------------------------
%\tableofcontents
%\cleardoublepage
%% 
%% % -----------------------------------------------------------------------------
%\chapter{Introduction to the Hazard Modeller's Toolkit}
%\begin{myfancybox}
%The objectives of this chapter are:
%\begin{itemize}
%    \item Outline the purpose and creation of the Hazard Modeller's Toolkit 
%    \item Setup and installation of the software
%    \item Introduction to the visualisation and mapping functions
%\end{itemize}
%\end{myfancybox}
%  The Hazard Modeller's Toolkit (or ``openquake.hmtk'') is a Python library of functions originally written by scientists at the GEM Model Facility, and now maintained by the GEM Foundation Secretariat. The HMTK is intended to provide 
scientists and engineers with the tools to help create the seismogenic 
input models that go into the OpenQuake hazard engine. The process of 
developing a hazard model is a complex and often challenging 
one, and while many aspects of the practice are relatively common, the 
choice of certain methods or tools for undertaking each step can be a 
matter of judgement. The intention of this software is to provide 
scientists and engineers with the means to apply many of the most 
commonly used algorithms for preparing seismogenic source models 
using seismicitiy and geological data. 

This manual is Version 2.0 of the HMTK tutorial. The major differences in the toolkit and the tutorial compared to the original release are i) the HMTK is now contained in the OpenQuake Engine, and does not require any separate installation, ii) the OpenQuake $hazardlib$ source classes have been adopted in order to ensure full compatibility and consistency between the two libraries, and iii) the plotting functions that produce maps now use Generic Mapping Tools (GMT) and Python scripts housed in the OpenQuake Model Building Toolkit. 


\section{The Development Process}

The Hazard Modeller's Toolkit is developed by GEM, and has occurred in several different stages. The present version makes the modelling tools available as a library, reflecting the general trend in the OpenQuake development process toward having a modular software framework. This means that the modelling - hazard - risk process is separated into libraries (e.g. oq-hazardlib, oq-risklib) that can be utilised as standalone tools, in addition to being integrated within the OpenQuake engine and platform. This is designed to allow for flexibility in the process, and also allow the user to begin to utilise (possibly in other contexts) functions and classes that are intended to address particular stages of the calculation. Such an approach ensures that each sub-component of the toolkit is fully tested, with a minimal degree of duplication in the testing process. In the HMTK this is taken a step further, as we are aiming to provide the hazard modeller as much control over the modelling process as possible, while retaining as complete a level of code testing as is practical to implement given the development resources available. 

The HMTK aims to address particular objectives:

\begin{description}
\item[Portability] Reduction in the number of Python dependencies to allow for a high degree of cross-platform deployment 

\item[Adaptability] Cleaner separation of methods into self-contained components that can be implemented and tested within requiring adaption of the remainder of the code.

\item[Abstraction] This concept is often a critical component object-oriented development. It describes the specification of a core behaviour of a method, which implementations (by means of the subclass) must follow. For example, a declustering algorithm must follow the common behaviour path, in this instance i) reading and earthquake catalogue and some configurable parameters, ii) identifying the clusters of events, iii) identifying the mainshocks from within each cluster,iv) returning this information to the user. The details of the implementation are then dependent on the algorithm, providing that the core flow is met. This is designed to allow the algorithms to be \emph{interchangeable} in the sense that different methods for  particular task could be selected with no (or at least minimal) modification to the rest of the code.

\item[Usability] The creation of a library which could itself be embedded within larger applications (e.g. as part of a graphical user interface).
 
\end{description}



\section{Getting Started and Running the Software}

The Modeller's Toolkit and associated software are designed for execution 
from the command line. As with the OpenQuake Engine, the preferred environment is 
Ubuntu Linux (12.04 or later), but is also supported on other operating systems.
Since the HMTK is contained by the OpenQuake Engine, all dependencies required by the HMTK itself are installed alongside the OpenQuake Engine. For more information regarding the current dependencies and installing the OpenQuake Engine, see \href{https://github.com/gem/oq-engine}{https://github.com/gem/oq-engine}.


\subsection{Current Features}

The Hazard Modeller's Toolkit is currently divided into three sections: 

\begin{enumerate}
\item \textbf{Earthquake Catalogue and Seismicity Analysis}
    These functions are intended to address the needs of defining seismic activity rate from an earthquake catalogue. They algorithms for identification of Non-Poissonian events (declustering), analysis of catalogue completeness, calculation of activity rate and b-value and, finally, estimation of maximum magnitude using statistical analyses of the earthquake catalogue. Also included in these tools is an initial implementation of a smoothed seismicity algorithm using the \textcite{frankel1995} approach.
     
\item \textbf{Active Faults Source Models from Geological Data}

    These functions are intended to address the Modeller needs for defining earthquake activity rates on fault sources from the geological slip rate, including support for some epistemic uncertainty analysis on critical parameters in the process.

\item \textbf{Seismic Source Models from Geodetic Data}

    These functions are intended to address the use of geodetic data to derive seismic activity rates from a strain rate model for a region, implementing the Seismic Hazard Inferred from Tectonics (SHIFT) methodology developed by \textcite{BirdLiu2007} and applied on a global scale by \textcite{Bird_etal2010}.
\end{enumerate}

A summary of the algorithms available in the present version is given in Table \ref{tab:current_features}.
\begin{table}
\centering
\begin{tabular}{|c|c|} \hline
\textbf{Feature} & \textbf{Algorithm}\\ \hline
\textbf{Seismicity} & \\ \hline
Declustering & \textcite{GardnerKnopoff1974}  \\
    & AFTERAN \parencite{Musson1999} \\ \hline
Completeness & \textcite{Stepp1971}\\ \hline
Recurrence & Maximum Likelihood \parencite{Aki1965}\\
 & Time-dependent MLE\\
 & \textcite{Weichert1980}\\ \hline
 Smoothed Seismicity & \textcite{frankel1995} \\ \hline
 \textbf{Geology} & \\ \hline
 Recurrence & \textcite{AndersonLuco1983} ``Arbitrary''\\
  & \textcite{AndersonLuco1983} ``Area $M_{MAX}$''\\
  & Characteristic (Truncated Gaussian) \\
  & \textcite{YoungsCoppersmith1985} Exponential\\
  & \textcite{YoungsCoppersmith1985} Characteristic\\ \hline
 \textbf{Geodetic Strain} & \\ \hline
 Recurrence & Seismic Hazard Inferred from Tectonics (SHIFT) \\
           &  \textcite{BirdLiu2007, Bird_etal2010} \\ \hline
\end{tabular}
\caption{Current algorithms in the HMTK}
\label{tab:current_features}
\end{table}

\subsection{About this Tutorial}

As previously indicated, the Modeller's Toolkit itself is a Python library. This means that its functions can be utilised in many different python applications. It is not, at present, a stand-alone software, and requires some investment of time from the user to understand the functionalities and learn how to link the various tools together into a workflow that will be suitable for the modelling problem at hand.

This manual is designed to explain the various functions in the toolkit and to provide some illustrative examples showing how to implement them for particular contexts and applications. The tutorial itself does not specifically require a working knowledge of Python. However, an understanding of the basic python data types, and ideally some familiarity with the use of Python objects, is highly desirable. Users who are new to Python are recommended to familiarise themselves with Appendix \ref{sec:python_guide} of this tutorial. This provides a brief overview of the Python programming language and should introduce concepts such as classes and dictionaries, which will be encountered in due course. For more detail of the complete Python language, a comprehensive overview of its features and usage standard python documentation (\href{http://docs.python.org/2/tutorial/}{http://docs.python.org/2/tutorial/}). Where necessary particular Python programming concepts will be explained in further detail.

The code snippets (indicated by verbatim text) can be executed from within an ''Interactive Python (IPython)'' environment, or may form the basis for usage of the openquake.hmtk in other python scripts that the user may wish to run construct themselves. If not already installed on your system, IPython can be installed from the python package repository by entering: 

\begin{Verbatim}[frame=single, commandchars=\\\{\}, fontsize=\scriptsize]
~\$ sudo pip install ipython
\end{Verbatim}

An ``interactive'' session can then be opened by typing \verb=ipython= at the command prompt. If \verb=matplotlib= is installed and you wish to use the plotting functionalities described herein then you should open IPython with the command:

\begin{Verbatim}[frame=single, commandchars=\\\{\}, fontsize=\scriptsize]
~\$ ipython --pylab
\end{Verbatim}

To exit an IPython session at any time simply type \verb=exit=.

For a more visual application of the openquake.hmtk the reader is encouraged to utilise the ``IPython Notebook'' (\href{http://ipython.org/notebook.html}{http://ipython.org/notebook.html}). This novel tool implements IPython inside a web-browser environment, permitting the user to create and store real Python workflows that can be retrieved and executed, whilst allowing for images and text to be embedded. A screenshot of the openquake.hmtk used in an IPython Notebook environment is shown in Figure \ref{fig:notebook}. From version 1.0 of IPython, the IPython Notebook comes installed. A notebook session can be
started via the command:

\begin{Verbatim}[frame=single, commandchars=\\\{\}, fontsize=\scriptsize]
~\$ ipython notebook --pylab inline
\end{Verbatim}

\begin{figure}[htb]
  \centering
      \includegraphics[width=\textwidth]{./figures_v2/hmtk_notebook_screenshot.jpg}
  \caption{Example of the openquake.hmtk embedded in an IPython Notebook}
  \label{fig:notebook}
\end{figure}


\subsection{Visualisation}

In addition to the scientific tools, which will be described in detail in due course, the original version of the openquake.hmtk also included a set of functionalities for visualisation of data and results pertinent to the preparation of seismic hazard input models. While not considered an essential component of the openquake.hmtk, the usage of the plotting functions can facilitate model development. Particular visualisation functions shall be referred to where relevant for the particular tool or data set. 

The current version of the HMTK includes most of the original visualisation tools. However, the tools for map creation have been depracated, and replaced by a set of mapping functions in the \href{https://github.com/GEMScienceTools/oq-mbtk/tree/master/openquake}{OpenQuake Model Building Toolkit (MBTK)}. The tools now use \href{https://www.generic-mapping-tools.org}{Generic Mapping Tools (GMT)}, and so were moved outside of the HMTK library as to not add GMT as a dependency of the OpenQuake Engine. However, the mapping functions are still described in this tutorial in order to provide users with a replacement to the depracated functions. 

\subsubsection{Mapping tools: additional setup}
Within the plotting tools is a set of methods to create maps of geospatial data; these tools are housed in the MBTK. Use of the mapping tools requires the following additional package installations:\\

\begin{enumerate}
	\item \href{https://www.generic-mapping-tools.org/download/}{GMT} version 6.0 or later.
	\item \href{https://github.com/GEMScienceTools/oq-mbtk/tree/master/openquake}{MBTK}\\
\end{enumerate}

The functions of \cprotect{\href{https://github.com/GEMScienceTools/oq-mbtk/tree/master/openquake/plt}}{\verb=openquake.plt.mapping=} can also be used independently of the MBTK (NB: the descriptions herein assume that mapping functions are called through the MBTK). 

\subsubsection{Map Creation}
An IPython Notebook demonstrating how to use the mapping methods can be found \href{https://github.com/GEMScienceTools/oq-mbtk/tree/master/openquake/plt/demo}{in the MBTK}. The basic functionalities are described herein.

To set-up a simple basemap it is necessary to define the configuration of the plot (such as spatial limit and coastline resolution). This is done as follows:

\begin{python}[frame=single]
In [1]: from openquake.plt.mapping import HMTKBaseMap

In [2]: map_config = {"min_lon": 18.0,
                      "max_lon": 32.0,
                       "min_lat": 33.0,
                       "max_lat": 43.0,
		       "title": "Title of Map"}

In [3]: basemap1 = HMTKBaseMap(map_config)
\end{python}

\verb=HMTKBaseMap= is instantiated with a dictionary of configuration parameters: minimum longitude (\verb=min_lon=), maximum longitude (\verb=max_lon=), minimum latitude (\verb=min_lat=), maximum latitude (\verb=max_lat=). The map title (\verb=title=) can also be specified.

A few other configurations can be passed to \verb=HMTKBaseMap= via keyword parameters during the map instantiation:

\begin{itemize}
	\item \verb=projection=: String beginning with '-J' that indicates the map projection, and optionally central meridian and scaling, following the GMT syntax (\href{http://gmt.soest.hawaii.edu/doc/latest/gmt.html\#j-full}{GMT Map Projections}). The default `-JM15c' is a Mercator projection 15 cm wide.
\item \verb=lat_lon_spacing=: Indicates the spacing of latitude and longitude tickmarks. The default is 2 degrees.
\item \verb=output_folder=: Denotes the output directory for the final map and associated files (if saved, see \verb=.savemap=) relative to the local path. The \verb=output_folder= is immeidately created by \verb=HMTKBaseMap=, and used for all temporary files created during the mapping process. The default is \textit{gmt}. 
\item \verb=overwrite=: If True, gives permission to overwrite all existing files in the specified \verb=output_folder=.\\
\end{itemize}

\noindent The class \verb=HMTKBaseMap= contains a set of methods for mapping catalogue data or simplified source models:\\

\noindent \verb;.add_catalogue(cat, scale=0.05, cpt_file=`tmp.cpt', color_field=`depth',;\\
		\verb;logscale=True);\\

\noindent This function will overlay an earthquake catalogue onto the basemap. The input value \verb=cat= is the earthquake catalogue as an instance of the class \\\verb=openquake.hmtk.seismicity.catalogue.Catalogue= (see the next section for details). The catalogue is the only mandatory parameter, but the user can also specify the following optional parameters:\\

\begin{itemize}
	\item \verb=scale=: a scaling coefficient that sets the symbol size per magnitude $m$. Size follows the equation ${\verb=scale=}*10^{(-1.5+m*0.3)}$, where m is magnitude. See GMT documentation.
	\item \verb=cpt_file=: name of an existing color pallet to color earthquake markers. If not specified, the default "tmp.cpt" is generated based on the catalogue \verb=color_field=
	\item \verb=color_field=: the parameter used to color the earthquake markers. The given field must correspond to the catalogue header. If not specified, the markers are colored by depth.
	\item \verb=logscale=: if `True', generates the color pallet according to a log scale. `False' uses a linear color scale. Default is `True'. Ignored if \verb=cpt_file= is specified.\\
\end{itemize}


\begin{figure}[htb]
  \centering
      \includegraphics[width=\textwidth]{./figures_v2/catalogue.jpg}
      %\includegraphics[trim=20mm 14mm 1mm 1mm, clip, width=\textwidth]{./figures_v2/catalogue.jpg}
  \caption{Example visualisation of an Earthquake Catalogue}
  \label{fig:eqcat_simple}
\end{figure}


\noindent \verb;.add_source_model(model);\\

\noindent This method adds a source model to the basemap. The input value \verb=model= is an instance of the class \verb=openquake.hazardlib.nrml.SourceModel= (see section \textbf{TODO -> this is replacing the mtk source classes}). An example of a source model plot is shown in \ref{fig:source_model_map}.

NB: At present, only the following source typologies can be plotted automatically:

\begin{itemize}
\item Point sources
\item Simple faults
\item Complex faults
\item Area sources\\
\end{itemize}

Non-parametric sources and multi-point sources will be added soon.\\
 
\begin{figure}[htb]
  \centering
      \includegraphics[width=\textwidth]{./figures_v2/PNGSourceModel.jpg}
	\caption{Example visualisation of a source model for Papua New Guinea \textcite{ghasemi2016} with area sources (blue) and a complex fault.}
  \label{fig:source_model_map}
\end{figure}


\noindent \verb;.add_colour_scaled_points(longitude, latitude, data, label=`', shape=`-Ss', ;\\
\verb;                                    size=0.3, logscale=False);\\

\noindent This method overlays a set of data points with colour scaled according to the \verb=data= values. Three data arrays are required: one each with the \verb=longitude= and \verb=latitude= coordinates of the data points, and \verb=data=, a set of scalar values (e.g. magnitude or depth, if plotting an earthquake catalogue) associated with those points. In addition to these, the method takes four optional keyword parameters:\\
\begin{itemize}
\item \verb=label=: a string used to label the color scale; corresponds to \verb=data=
\item \verb=shape=: a string indicating the shape of the data markers, using GMT syntax starting with `-S' (see \href{https://docs.generic-mapping-tools.org/latest/psxy.html\#s}{GMT psxy markers}. The default, `-Ss' is squares.
\item \verb=size=: the size in cm of the plotted markers (see \href{https://docs.generic-mapping-tools.org/latest/psxy.html\#s}{GMT psxy markers}). Default is 0.3 cm.
\item \verb=logscale=: if True, use a logscale to create the colorbar. Default is False.\\
\end{itemize}

\begin{figure}[htb]
  \centering
      \includegraphics[width=\textwidth]{./figures_v2/colorscaled.jpg}
	\caption{Example of a seismicity catalogue with color scaled by magnitude.}
  \label{fig:cat_color_scaled}
\end{figure}


\noindent \verb;.add_size_scaled_points(longitude, latitude, data, shape=`-Ss',;\\
\verb;                       logplot=False, color=`blue', smin=0.01, coeff=1.0, ;\\
\verb;                       sscale=2.0, label=`', legend=True);\\

\noindent This method overlays a set of data points with size scaled according to the \verb=data= values. Three data arrays are required: one each with the \verb=longitude= and \verb=latitude= coordinates of points to be plotted, and \verb=data=, a set of scalar values associated with those points. In addition to these, the method takes eight optional keyword parameters:\\

\begin{itemize}
\item \verb=shape=: a string indicating the shape of the data markers, using GMT syntax starting with `-S' (see \href{https://docs.generic-mapping-tools.org/latest/psxy.html\#s}{GMT psxy markers}. The default, `-Ss' is squares.
\item \verb=logplot=: if True, use a logscale to create the marker sizes. Default is False.
\item \verb=color=: a string that indicates the marker color (see \href{https://docs.generic-mapping-tools.org/latest/psxy.html\#w}{GMT psxy markers}. Default is `blue'.
\item \verb=smin=: size of the smallest symbol in cm. Marker size is computed as\\ ${\verb=smin=}+{\verb=coeff=}\times {\verb=data=}^{\verb=sscale=}$. Default is 0.01 cm.
\item \verb=coeff=: used with \verb=sscale= and \verb=smin= to set the marker sizes. Default is 1.0.
\item \verb=sscale=: used with \verb=coeff= and \verb=smin= to set the marker sizes. Default is 2.0.
\item \verb=label=: a string that corresponds to the \verb=data= array
\item \verb=legend=: if True, adds a legend to the plot. Default is True. \\
\end{itemize}

\begin{figure}[htb]
  \centering
      \includegraphics[width=\textwidth]{./figures_v2/sizescaled.jpg}
	\caption{Example of a seismicity catalogue with size scaled by magnitude.}
  \label{fig:source_model_map}
\end{figure}

\noindent \verb;.add_focal_mechanism(filename, mech_format);\\

\noindent This method overlays focal mechanisms. The string \verb=filename= indicates a file containing focal mechanism data. \verb=mech_format= is a string contained by quotations used to indicate the data format used by \verb=filename=, allowing two options---focal mechanism (`FM') and seismic moment tensor (`MT')---both using the Harvard CMT convention, as described by \href{https://docs.generic-mapping-tools.org/latest/supplements/seis/psmeca.html?highlight=psmeca\#s}{GMT psmeca}.\\

\noindent \verb;.savemap(filename=None, save_script=False, verb=False);\\

An instance of \verb=HMTKBaseMap= is not automatically saved. In order to do so, the method \verb=.savemap()= must be called, finalizing and executing the GMT script. The method can take the following three keyword arguments:\\

\begin{itemize}
\item \verb=filename=: a string used to name the map, which includes a suffix (limited to `.pdf', `.png', and `.jpg') indicating the desired file type. If not specified, the map is saved as $map.pdf$ in the directory \verb=output_folder= that was assigned during the \verb=HMTKBaseMap= instantiatation.
\item \verb=save_script=: if True, the GMT commands are saved to a shell script, and this with all files needed to create the map are saved in \verb=output_folder=. If False (the default), all the temporary files are erased and only the map is saved.
\item \verb=verb= (verbose): if True, GMT commands are printed as they are executed.\\
\end{itemize}

The \verb=save_script= option gives the user more flexibility to modify the plot settings than are available through the methods, while providing the structure of the GMT script as a starting point. NB: Take care not to overwrite scripts that have been customized by rerunning the mapping code! \\

The \verb=.savemap()= method is used as follows (continuing from the above Python lines):\\

\begin{python}[frame=single]
In [4]: finame = 'map_demo.pdf'

In [5]: basemap1.savemap(filename=finame, save_script=True)
\end{python}


%\cleardoublepage
%% % -----------------------------------------------------------------------------
%% % -----------------------------------------------------------------------------
%
%\chapter{Seismicity Tools}
%\begin{myfancybox}
%The objectives of this chapter are:
%\begin{itemize}
%    \item Describe features of the seismicity tools of the Hazard Modeller's Toolkit 
%    \item Run simple calculations using the seismicity tools
%\end{itemize}
%\end{myfancybox}
%  \section{The Earthquake Catalogue}

The seismicity tools are intended for use in deriving activity rates from an observed earthquake catalogue, which may include both instrumental and historical seismicity. The tools are broken down into five separate libraries: i) Declustering, ii) Completeness, iii) Calculation of Gutenberg-Richter a- and b-value, iv) Statistical estimators of maximum magnitude from seismicity) and v) Smoothed Seismicity. In a common use case it is likely that many of the above methods, particularly recurrence and maximum magnitude estimation, may need to be applied to a selected sub-catalogue (e.g. earthquakes within a particular polygon). The toolkit allows for the creation of a source model containing one or more of the supported OpenQuake seismogenic source typologies, which can be used as a reference for selection, e.g. events within an area source (polygon), events within a distance of a fault etc. The supported input formats for both the catalogue are described below, and the source models in the subsequent chapter. 

\subsection{The Catalogue Format and Class}

The input catalogue must be formatted as a comma-separated value file (.csv), with the following attributes in the header line (attributes with an * indicate essential attributes), although the order of the columns need not be fixed:

\begin{table}
\begin{tabular}{|l|l|}  \hline 
Attribute & Description \\ \hline
eventID* & A unique identifier (integer) for each earthquake in the catalogue \\
Agency & The code (string) of the recording agency for the event solution  \\
year* & Year of event (integer) in the range -10000 to present \\
 & (events before common era (BCE) should have a negative value)\\
month* & Month of event (integer)\\
day* & Day of event (integer) \\
hour* & Hour of event (integer) - if unknown then set to 0 \\
minute* & Minute of event (integer) - if unknown then set to 0 \\
second* & Second of event (float) - if unknown set to 0.0 \\
timeError & Error in event time (float) \\
longitude* & Longitude of event, in decimal degrees (float) \\
latitude* & Latitude of event, in decimal degrees (float) \\
SemiMajor90 & Length (km) of the semi-major axis of the 90 \% \\
            & confidence ellipsoid for location error (float) \\
SemiMinor90 & Length (km) of the semi-minor axis of the 90 \% \\
            & confidence ellipsoid for location error (float) \\
ErrorStrike & Azimuth (in degrees) of the 90 \% \\
            & confidence ellipsoid for location error (float) \\
depth* & Depth (km) of earthquake (float)\\
depthError & Uncertainty (as standard deviation) in earthquake depth (km) (float)\\
magnitude* & Homogenised magnitude of the event (float) - typically Mw \\
sigmaMagnitude* & Uncertainty on the homogenised magnitude (float) typically Mw \\ \hline
\end{tabular}
\caption{List of Attributes in the Earthquake Catalogue File (* Indicates Essential)}
\label{tab: EQCatalogueFormat}
\end{table}

To load the catalogue using the IPython environment, in an open IPython session type:

%\begin{Verbatim}[frame=single, commandchars=\\\{\}, fontsize=\scriptsize, samepage=true]
\begin{python}
>> from openquake.hmtk.parsers.catalogue import CsvCatalogueParser
>> catalogue_filename = 'path/to/catalogue_file.csv'
>> parser = CsvCatalogueParser(catalogue_filename)
>> catalogue = parser.read_file()
\end{python}

\textbf{N.B. the csv file can contain additional attributes of the catalogue too and will be parsed correctly; however, if the attribute is not one that is specifically recognised by the catalogue class then a message will be displayed indicating:}

\begin{Verbatim}[frame=single, commandchars=\\\{\}, fontsize=\scriptsize, samepage=true]
Catalogue Attribute ... is not a recognised catalogue key 
\end{Verbatim}

\textbf{This is expected behaviour and simply indicates that although this data is given in the input file, it is not retained in the data dictionary.}

The variable \verb=catalogue= is an instance of the class openquake.hmtk.seismicity.catalogue.Catalogue, which now contains the catalogue itself (as \verb=catalogue.data=) and some methods that can be applied to the catalogue. The first attribute (\verb=catalogue.data=), is a dictionary where each attribute of the catalogue is either a 1-D numpy vector (for float and integer values) or a python list (for string values). For example, to return a vector containing all the magnitudes in the \verb=magnitude= column of the catalogue simply type:

\begin{python}
>> catalogue.data['magnitude']
array([ 6.5,  6.5,  6. , ...,  4.8,  5.2,  4.1])
\end{python}

The catalogue class contains several helpful methods (called via \verb=catalogue. ...=):
\begin{itemize}
\item \verb=catalogue.get_number_events()= Returns the number of events currently in the catalogue (integer)

\item \verb=catalogue.load_to_array(keys)= Returns a numpy array of floating data, with the columns ordered according to the list of keys. If the key corresponds to a string item (e.g. Agency) then an error will be raised.

\begin{python}[frame=single]
>> catalogue.load_to_array(['year', 'longitude', 'latitude',
                            'depth', 'magnitude'])
array([[ 1910. ,  26.941 ,  38.507 ,  13.2 ,  6.5 ],
       [ 1910. ,  22.190 ,  37.720 ,  20.4 ,  6.5 ],
       [ 1910. ,  28.881 ,  33.274 ,  25.0 ,  6.0 ],
       ..., 
       [ 2009. ,  20.054 ,  39.854 ,  20.2 ,  4.8 ],
       [ 2009. ,  23.481 ,  38.050 ,  15.2 ,  5.2 ],
       [ 2009. ,  28.959 ,  34.664 ,  18.4 ,  4.1 ]]) 
\end{python}

\item \verb=catalogue.load_from_array(keys, data_array)= Creates the catalogue data dictionary from an array, given header as an ordered list of dictionary keys. This can be used in the case where the earthquake catalogue is loaded in a simple ascii format. For example, if the user wishes to load in a catalogue from the Zmap format, which gives the columns as:

\begin{verbatim}
longitude, latitude, year, month, day, magnitude, depth, hour, 
minute, second
\end{verbatim}

This file type could be parsed into a catalogue without the need of a specific parser, as follows:

\begin{python}[frame=single]
>> import numpy
# Assuming no headers in the file 
# (set skip_header=1 if headers are found)
>> data = numpy.genfromtxt('PATH/TO/ZMAP_FILE.txt',
                           skip_header=0)

>> headers = ['longitude', 'latitude', 'year', 'month',
              'day', 'magnitude', 'depth', 'hour', 
              'minute', 'second']

# Create instance of a catalogue class
>> from openquake.hmtk.seismicity.catalogue import Catalogue
>> catalogue = Catalogue()

# Load the data array into the catalogue
>> catalogue.load_from_array(data, headers)
\end{python}
 

\item \verb=catalogue.get_decimal_time()= 

Returns the time of the earthquake in a decimal format

\item \verb=catalogue.hypocentres_as_mesh()=

Returns the hypocentres of an earthquake as an instance of the class \\ ``openquake.hazardlib.geo.mesh.Mesh'' (useful for geospatial functions)

\item \verb=catalogue.hypocentres_to_cartesian()=

Returns the hypocentres in a 3D cartesian framework

\item \verb=catalogue.purge_catalogue(flag_vector)=

Purges the catalogue of all \verb=False= events in the boolean vector. Thus is used for removing foreshocks and aftershocks from a catalogue after the application of a declustering algorithm.

\item \verb=catalogue.sort_catalogue_chronologically()=

Sorts an input into chronological order. \\
\emph{N.B. Some methods will implicitly assume that the catalogue is in chronological order, so it is recommended to run this function if you believe that there may be events out of order}

\item \verb=catalogue.select_catalogue_events(IDX)=

Orders the catalogue according to the event order specified in IDX. Behaves the same as \verb=purge_catalogue(IDX)= if IDX is a boolean vector

\item \verb;catalogue.get_depth_distribution(depth_bins, normalisation=False,;\\
\verb;    bootstrap=None);

Returns a depth histogram for the catalogue using bins specified by \verb=depth_bins=. If \verb;normalisation=True; then the function will return the histogram as a probability mass function, otherwise the original count will be returned. If uncertainties are reported on depth such that one or more values in \\ \verb=catalogue.data['depthError']= are greater than 0., the function will perform a bootstrap analysis, taking into account the depth error, with the number of bootstraps given by the keyword \verb=bootstrap=. 

\begin{python}[frame=single]
# Import numpy and matplotlib
>> import numpy as np
>> import matplotlib.pyplot as plt

# Define depth bins for (e.g) 
# 0. - 150 km in intervals of 10 km
>> depth_bins = np.arange(0., 160., 10.)

# Get normalised histograms (without bootstrapping)
>> depth_hist = catalogue.get_depth_distribution(
    depth_bins,
    normalisation=True)
\end{python}

To generate a simple histogram plot of hypocentral depth, the process below can be followed to produce a depth histogram similar to the one shown in Figure \ref{fig:simple_depth_hist}:

\begin{python}[frame=single]
>> from openquake.hmtk.plotting.seismicity.catalogue_plots import\
    plot_depth_histogram

>> depth_bin = 5.0
>> plot_depth_histogram(catalogue,
                        depth_bin,
                        filename="/path/to/image.eps",
                        filetype="eps")
\end{python}
\begin{figure}[htb]
  \centering
      \includegraphics[trim=10mm 8mm 10mm 10mm, clip, width=12cm]{./figures/simple_depth_histogram.eps}
  \caption{Example depth histogram}
  \label{fig:simple_depth_hist}
\end{figure}

\item \verb;catalogue.get_magnitude_depth_distribution(magnitude_bins, depth_bins,;\\
\verb;    normalisation=False, bootstrap=None); 

Returns a two-dimensional histogram of magnitude and hypocentral depth, with the corresponding bins defined by the vectors \verb=magnitude_bins= and \verb=depth_bins=. The options \verb=normalisation= and \verb=bootstrap= are the same as for the one dimensional histogram. The usage is illustrated below:

\begin{python}[frame=single]]
# Define depth bins for (e.g) 
# 0. - 150 km in intervals of 55 km
>> depth_bins = np.arange(0., 155., 5.)

# Define magnitude bins (e.g.) 2.5 - 7.6 in intervals of 0.1
>> magnitude_bins = np.arange(2.5, 7.7, 0.1)

# Get normalised histograms (without bootstrapping)
>> m_d_hist = catalogue.get_magnitude_depth_distribution(
    magnitude_bins,
    depth_bins,
    normalisation=True,
    bootstrap=None)
\end{python}

To generate a plot of magnitude-depth density, the following function can be used to produce a figure similar to that shown in Figure \ref{fig:mag_depth_density}.

\begin{python}[frame=single]
>> from openquake.hmtk.plotting.seismicity.catalogue_plots import\
     plot_magnitude_depth_density
>> magnitude_bin = 0.1
>> depth_bin = 5.0 
>> plot_magnitude_depth_density(
    catalogue,
    magnitude_bin
    depth_bin,
    logscale=True, \# Logarithmic colour scale
    filename="/path/to/image.eps", \# Optional
    filetype="eps")   \# Optional
\end{python}

\begin{figure}[htb]
  \centering
      \includegraphics[trim=10mm 10mm 10mm 10mm, clip, width=14cm]{./figures/magnitude_depth_density.eps}
  \caption{Example magnitude-depth density plot}
  \label{fig:mag_depth_density}
\end{figure}

\item \verb;catalogue.get_magnitude_time_distribution(magnitude_bins, time_bins,;\\
\verb; normalisation=False, bootstrap=None);

Returns a 2D histogram of magnitude with time. \verb=time_bins= are the bin edges for the time windows, in decimal years. To run the function simple follow:

\begin{python}[frame=single]
# Define annual time bins from 1900 CE to 2012 CE
>> time_bins = np.arange(1900., 2013., 1.)
# Define magnitude bins (e.g.) 2.5 - 7.6 in intervals of 0.1
>> magnitude_bins = np.arange(2.5, 7.7, 0.1)
# Get normalised histograms (without bootstrapping)
>> mag_time_hist = catalogue.get_magnitude_time_distribution(
    magnitude_bins,
    time_bins,
    normalisation=True,
    bootstrap=None)
\end{python}

To automatically generate a plot, similar to that shown in Figure \ref{fig:mag_time_density} , run the following:

\begin{python}[frame=single]
>> from openquake.hmtk.plotting.seismicity.catalogue_plots import\
    plot_magnitude_time_density
>> magnitude_bin_width = 0.1
>> time_bin_width = 0.1 
>> plot_magnitude_time_density(catalogue,
                               magnitude_bin_width,
                               time_bin_width,
                               filename="/path/to/image.eps", 
                               filetype="eps")
\end{python}

\begin{figure}[htb]
  \centering
      \includegraphics[trim=10mm 8mm 10mm 10mm, clip, width=12cm]{./figures/magnitude_time_density.eps}
  \caption{Example magnitude-time density plot}
  \label{fig:mag_time_density}
\end{figure}
\end{itemize}

\subsection{The ``Selector'' Class}

In the process of constructing a PSHA seismogenic source model from seismicity it is necessary to select sub-sets of the earthquake catalogue, usually for calculating earthquake recurrence statistics pertinent to a particular region or seismogenic source. As catalogue selection is such a prevalent aspect of the source modelling process, the selection is done inside the HMTK via the use of a "Selector" tool. This tools is a container for all methods associated with the selection of sub-catalogues from a given earthquake catalogue. It will be seen in due course that later methods relating to the selection of the catalogue for a particular source require as an input an instance of the selector class, rather than the catalogue itself.

To setup the ``Selector'' tool:

\begin{python}[frame=single]
>> from openquake.hmtk.seismicity.selector import CatalogueSelector

# Assuming that there already exists a 
# catalogue named ''catalogue1''

>> selector1 = CatalogueSelector(catalogue1,
                                 create_copy=True)

\end{python}

The optional keyword \verb=create_copy= ensures that when the events not selected are purged from the catalogue a ``deepcopy'' is taken of the original catalogue. This ensures that the original catalogue remains unmodified when a subset of events is selected.

The catalogue selector class has the following methods:

\verb;.within_polygon(polygon, distance=None);

Selects events within a polygon described by the class \verb=openquake.hazardlib.geo.=\\\verb=polygon.Polygon=. \verb=distance= is the distance (in km) to use as a buffer, if required. Optional keyword arguments \verb=upper_depth= and \verb=lower_depth= can be used to limit the depth range of the catalogue returned by the selector to only those events whose hypocentres are within the specified depth limits.

\verb;.circular_distance_from_point(point, distance, distance_type="epicentral");

Selects events within a distance from the a location. The location (\verb=point=) is an instance of the openquake.hazardlib.geo.point.Point class, whilst \verb=distance= is the selection distance (km) and \verb=distance_type= can be either "epicentral" or "hypocentral".  

\verb;.cartesian_square_centred_on_point(point, distance);

Selects events within a square of side length \verb=distance=, on a location (represented as an openquake \verb=Point= class).

\verb;.within_joyner_boore_distance(surface, distance);

Returns earthquakes within a distance (km) of the surface projection (``Joyner-Boore'' distance) of a fault surface. The fault surface must be defined as an instance of the class \\\verb=openquake.hazardlib.geo.surface.simple_fault.SimpleFaultSurface= or\\ \verb=openquake.hazardlib.geo.surface.complex_fault.ComplexFaultSurface=.

\verb;.within_rupture_distance(surface, distance);

Returns earthquakes within a distance (km) of a fault surface. The fault surface must be defined as an instance of the class \\\verb=openquake.hazardlib.geo.surface.simple_fault.SimpleFaultSurface= or\\ \verb=openquake.hazardlib.geo.surface.complex_fault.ComplexFaultSurface=.

\verb;.within_time_period(start_time=None, end_time=None);

Selects earthquakes within a time period. Times must be input as instances of a \verb=datetime= object. For example:

\begin{python}[frame=single]
>> from datetime import datetime
>> selector1 = CatalogueSelector(catalogue1, create_copy=True)
# Early time limit is 1 January 1990 00:00:00
>> early = datetime(1990, 1, 1, 0, 0, 0)
# Late time limit is 31 December 1999 23:59:59
>>: late = datetime(1999, 12, 31, 23, 59, 59)
>> catalogue_nineties = selector1.within_time_period(
    start_time=early,
    end_time=late)
\end{python}

\verb;.within_depth_range(lower_depth=None, upper_depth=None);

Selects earthquakes whose hypocentres are within the range specified by the lower depth limit (\verb=lower_depth=) and the upper depth limit (\verb=upper_depth=), both in km.

\verb;.within_magnitude_range(lower_mag=None, upper_mag=None);

Selects earthquakes whose magnitudes are within the range specified by the lower limit (\verb=lower_mag=) and the upper limit (\verb=upper_mag=).

\section{Declustering}

To identify Poissonian rate of seismicity, it is necessary to remove foreshocks/aftershocks/swarms from the catalogue. The Modeller's Toolkit contains, at present, two algorithms to undertake this task, with more under development.

\subsection{\textcite{GardnerKnopoff1974}}

The most widely applied simple windowing algorithm is that of 
\textcite{GardnerKnopoff1974}. Originally conceived for Southern California, 
the method simply identifies aftershocks by virtue of fixed time-distance
windows proportional to the magnitude of the main shock. Whilst this 
premise is relatively simple, the manner in which the windows are 
applied can be ambiguous. Four different possibilities can be 
considered \parencite{LuenStark2012}:

\begin{enumerate}
\item Search events in magnitude-descending order. Remove events if it is 
    in the window of the largest event
\item Remove every event that is inside the window of a previous event, 
    including larger events
\item An event is in a cluster if, and only if, it is in the window of at 
    least one other event in the cluster. In every cluster remove all 
    events except the largest
\item In chronological order, if the $i^{th}$ event is in the window of a 
    preceding larger shock that has not already been deleted, remove it. 
    If a larger shock is in the window of the $i^{th}$ event, delete the 
    $i^{th}$ event. Otherwise retain the $i^{th}$ event.
\end{enumerate}

It is the first of the four options that is implemented in the current 
toolkit, whilst others may be considered in future.  The algorithm is 
capable if identifying foreshocks and aftershocks, simply by applying 
the windows forward and backward in time from the mainshock. 
No distinction is made between primary aftershocks (those resulting 
from the mainshock) and secondary or tertiary aftershocks (those 
originating due to the previous aftershocks); however, it is assumed 
all would occur within the window.

Several modifications to the time and distance windows have been 
suggested, which are summarised in \textcite{vanStiphout2012}. The windows 
originally suggested by \textcite{GardnerKnopoff1974} are approximated by:

\begin{equation}\begin{split} 
\mbox{distance (km)} = &10^{0.1238 M + 0.983}\\
\mbox{time (decimal years)} = & 
\begin{cases} 10^{0.032 M + 2.7389} & \text{if $M \geq 6.5$} \\ 
              10^{0.5409 M - 0.547} & \mbox{otherwise}  \end{cases}\end{split}
\end{equation}

An alternative formulation is proposed by Gr\"unthal (as reported in \textcite{vanStiphout2012}):

\begin{equation}\begin{split} 
\mbox{distance (km)} = & e^{1.77 + \left( {0.037 + 1.02 M} \right)^2} \\ 
   \mbox{time (decimal years)} = & \begin{cases}   |e^{-3.95+ \left( {0.62 + 17.32 M}
    \right)^2}|    & \text{if $M \geq 6.5$ } \\ 10^{2.8 + 0.024 M} & 
    \text{otherwise}  \end{cases}\end{split}
\end{equation}
A further alternative is suggested by \textcite{Uhrhammer1986}
%
\begin{equation}
\mbox{distance (km)} = e^{-1.024 + 0.804 M} \quad \mbox{time (decimal years)} = 
    e^{-2.87 + 1.235 M}
\end{equation}

A comparison of the expected window sizes with magnitude are shown for 
distance  and time (Figure \ref{fig:declust_scaling}).

\begin{figure}[htb]
  \centering
  \begin{subcaption}
      \centering
      \includegraphics[width=8cm]{./figures/declustering_distance_windows.eps}
	\end{subcaption}
  \begin{subcaption}
      \centering
      \includegraphics[width=8cm]{./figures/declustering_time_windows.eps}
	\end{subcaption}	
	\caption{Scaling of declustering time and distance windows with magnitude}
	\label{fig:declust_scaling}
\end{figure}

The \textcite{GardnerKnopoff1974} algorithm and its derivatives represent 
are most computationally straightforward approach to declustering. The \verb=time_dist_windows= attribute indicates the choice of the 
time and distance window scaling model from the three listed. As 
the current version of this algorithm considers the events in a 
descending-magnitude order, the parameter \verb=foreshock_time_window= 
defines the size of the time window used for searching for foreshocks, 
as a fractional proportion of the size of the aftershock window (the 
distance windows are always equal for both fore- and aftershocks). 
So for an evenly sized time window for foreshocks and aftershocks,the\\
\verb=foreshock_time_window= parameter should equal 1. For shorter or longer 
foreshock time windows this parameter can be reduced or increased respectively.

To run a declustering analysis on the earthquake catalogue it is necessary to set-up the configuration using a python dictionary (see Appendix \ref{sec:python_guide}). A config file for the \textcite{GardnerKnopoff1974} algorithm, using for example the \textcite{Uhrhammer1986} time-distance windows with equal sized time window for aftershocks and foreshocks, would be created as shown:

\begin{python}[frame=single]
>> from openquake.hmtk.seismicity.declusterer.distance_time_windows import\
    UhrhammerWindow

>> declust_config = {
    'time_distance_window': UhrhammerWindow(),
    'fs_time_prop': 1.0}
\end{python}


To run the declustering algorithm simply import and run the algorithm as shown:

\begin{python}[frame=single]
>> from openquake.hmtk.seismicity.declusterer.dec_gardner_knopoff import\
    GardnerKnopoffType1

>> declustering = GardnerKnopoffType1()

>> cluster_index, cluster_flag = declustering.decluster(
    catalogue,
    declust_config)
\end{python}

There are two outputs of a declustering algorithm: \verb=cluster_index= and \verb=cluster_flag=. Both are numpy vectors, of the same length as the catalogue, containing information about the clusters in the catalogue. \verb=cluster_index= indicates the cluster to which each event is assigned (0 if not assigned to a cluster). \verb=cluster_flag= indicates whether an event is a non-Poissonian event, in which case the value is assigned to 1, or a mainshock, the value is assigned as 0. This output definition is the same for all declustering algorithms.

At this point the user may wish to either retain the catalogue in its current format, in which case they may wish to add on the clustering information into another attribute of the catalogue.data dictionary, or they may wish to purge the catalogue of non-Poissonian events. 

To simply add the clustering information to the data dictionary simply type:

\begin{python}[frame=single]
>> catalogue.data['Cluster_Index'] = cluster_index
>> catalogue.data['Cluster_Flag'] = cluster_flag
\end{python}
 
Alternatively, to purge the catalogue of non-Poissonian events:

\begin{python}[frame=single]
>> mainshock_flag = cluster_flag == 0
>> catalogue.purge_catalogue(mainshock_flag)
\end{python}


\subsection{AFTERAN \parencite{Musson1999PSHABalkan}}

A particular development of the standard windowing approach is introduced in the program AFTERAN \parencite{Musson1999PSHABalkan}. This is a modification of the \textcite{GardnerKnopoff1974} algorithm, using a moving time window rather than a fixed time window. In AFTERAN, considering each earthquake in order of descending magnitude, events within a fixed distance window are identified (the distance window being those suggested previously). These events are searched using a moving time window of T days. For a given mainshock, non Poissonian events are identified if they occur both within the distance window and the initial time window. The time window is then moved, beginning at the last flagged event, and the process repeated. For a given mainshock, all non-Poissonian events are identified when the algorithm finds a continuous period of T days in which no aftershock or foreshock is identified. 

The theory of the AFTERAN algorithm is broadly consistent with that of \textcite{GardnerKnopoff1974}. This algorithm, whilst a little more computationally complex, and therefore slower, than the \textcite{GardnerKnopoff1974} windowing approach, remains simple to implement. 

As with the \textcite{GardnerKnopoff1974} function, the \verb=time_dist_window= attribute indicates the choice of the time and distance window scaling model. The parameter \verb=time_window= indicates the size (in days) of the moving time window used to identify fore- and aftershocks. The following example will show how to run the AFTERAN algorithm, using the  \textcite{GardnerKnopoff1974} definition of the distance windows, and a fixed-width moving time window of 100 days.

  
\begin{python}[frame=single]

>> from openquake.hmtk.seismicity.declusterer.dec_afteran import\
    Afteran

>> from openquake.hmtk.seismicity.declusterer.distance_time_windows import\
    GardnerKnopoffWindow   
 
>> declust_config = {
    'time_distance_window': GardnerKnopoffWindow(),
    'time_window': 100.0} 

>> declustering = Afteran()

>> cluster_index, cluster_flag = declustering.decluster(
    catalogue,
    declust_config)
\end{python} 

%::::::::::::::::::::::::::::::::::::::::::::::::::::::::::::::::::::::::::::::::::::::::::::::::::::::::::::::::::::::::::::::::::::::::::::::::::::::::::::::

\section{Completeness}

In the earliest stages of processing an instrumental seismic catalogue to derive inputs for seismic hazard analysis, it is necessary to determine the magnitude completeness threshold of the catalogue. To outline the meaning of the term ''magnitude completeness'' and the requirements for its analysis as an input to PSHA, the terminology of \textcite{MignanWoessner2012} is adopted. This defines the magnitude of completeness as the ''lowest magnitude at which 100 \% of the events in a space-time volume are detected \parencite{RydelekSacks1989, WoessnerWiemer2005}''. Incompleteness of an earthquake catalogue will produce bias when determining models of earthquake recurrence, which may have a significant impact on the estimation of hazard at a site. Identification of the completeness magnitude of an earthquake catalogue is therefore a clear requirement for the processing of input data for seismic hazard analysis.

It should be noted that this summary of methodologies for estimating completeness is directed toward techniques that can be applied to a ''typical'' instrumental seismic catalogue. We therefore make the assumption that the input data will contain basic information for each earthquake such as time, location, magnitude. We do not make the assumption that network-specific or station-specific properties (e.g., configuration, phase picks, attenuation factors) are known a priori. This limits the selection of methodologies to those classed as estimators of ''sample completeness'', which defines completeness on the basis of the statistical properties of the earthquake catalogue, rather than ''probability-based completeness'', which defines the probability of detection given knowledge of the properties of the seismic network \parencite{SchorlemmerWoessner2008}. This therefore excludes the methodology of \textcite{SchorlemmerWoessner2008}, and similar approaches such as that of \textcite{Felzer2008}

The current workflows assume that completeness will be applied to the whole catalogue, ideally returning a table of time-varying completeness. The option to explore spatial variation in completeness is not explicitly supported, but could be accommodated by an appropriate configuration of the toolkit.

In the current version of the Modeller's Toolkit the \textcite{Stepp1971} methodology for analysis of catalogue completeness is implemented. Further methods are in development, and will be input in future releases.

%\subsection{User-defined Table}
%
%This is simply a filtering that will remove from further consideration any events outside of the completeness bounds defined by the user. The table represents the time variation in $M_C$ and can be input as a separate file (in comma-separated value format) in the following format.
%\begin{Verbatim}[frame=single, commandchars=\\\{\}, fontsize=\scriptsize]
%1990.0, 4.0\\
%1960.0, 5.0\\
%1900.0, 6.0\\
%1700.0, 7.0\\
%\end{Verbatim}
%
%The left-hand column represents the earliest year at which the earthquake is complete at the corresponding magnitude in the right-hand column. \\
%\textbf{Important: The values in the completeness file must be entered from most-recent to oldest!} 

\subsection{\cite{Stepp1971}}

This is one of the earliest analytical approaches to estimation of completeness magnitude. It is based on estimators of the mean rate of recurrence of earthquakes within given magnitude and time ranges, identifying the completeness magnitude when the observed rate of earthquakes above $M_C$ begins to deviate from the expected rate. If a time interval ($T_i$) is taken, and the earthquake sequence assumed Poissonian, then the unbiased estimate of the mean rate of events per unit time interval of a given sample is:

\begin{equation}
   \lambda = \frac{1}{n} \sum_{i = 1}^{n} T_i
\end{equation}

with variance $\sigma_{\lambda}^{2} = \lambda / n$. Taking the unit time interval to be 1 year, the standard deviation of the estimate of the mean is:

\begin{equation}
   \sigma_{\lambda} = \sqrt{\lambda} / \sqrt{T}
\end{equation}

where $T$ is the sample length. As the Poisson assumption implies a stationary process, $\sigma_{\lambda}$ behaves as $1/\sqrt{T}$ in the sub-interval of the sample in which the mean rate of occurrence of a magnitude class is constant. Time variation of $M_C$ can usually be inferred graphically from the analysis, as is illustrated in Figure \ref{fig:SteppFigExample1}. In this example, the deviation from the $1/\sqrt{T}$ line for each magnitude class occurs at around 40 years  for $4.5 < M < 5$, 100 years for $5.0  < M < 6.0$, approximately 150 years for $6.0 < M < 6.5$ and 300 years for $M > 6.5$. Knowledge of the sources of earthquake information for a given catalogue may usually be reconciled with the completeness time intervals.

\begin{figure}[htb]
	\centering
		\includegraphics[height=10cm, keepaspectratio=true]{./figures/C2Fig1SteppFig1.eps}
	\caption{Example of Completeness Estimation by the \textcite{Stepp1971} methodology}
	\label{fig:SteppFigExample1}
\end{figure}

The analysis of \textcite{Stepp1971} is a coarse, but relatively robust, approach to estimating the temporal variation in completeness of a catalogue. It has been widely applied since its development. The accuracy of the completeness magnitude depends on the magnitude and time intervals considered, and a degree of judgement is often needed to determine the time at which the rate deviates from the expected values. It has tended to be applied to catalogues on a large scale, and for relatively higher completeness magnitudes. 

To translate the methodology from a largely graphical methods into a computational method the completeness period needs to be identified by automatically identifying the point at which the gradient of the observed values decreases with respect to that expected from a Poisson process (see \ref{fig:SteppFigExample1}). In the implementation found within the current toolkit, the divergence point is identified by fitting a two-segment piecewise linear function to the observed data. Although a two-segment piecewise linear function is normally fit with four parameters (intercept, $slope_1$, $slope_2$ and crossover point), by virtue of the assumption that for the complete catalogue the rate is assumed to be stationary such that $\sigma_{\lambda} = \frac{1}{\sqrt{T}}$ the slope of the first segment can be fixed as $-0.5$, and the second slope should be constrained such that $slope_2 \leq -0.5$, whilst the crossover point ($x_c$) is subject to the constraint ($x_c \geq 0.0$). Thus it is possible to fit the two-segment linear function using constrained optimisation with only three free parameters. For this purpose the toolkit minimises the residual sum-of-squares of the model fit using numerical optimisation. 

To run the \textcite{Stepp1971} algorithm the configuration parameters should be entered in the form of a dictionary, such as the example shown below:

\begin{python}[frame=single]
comp_config = {'magnitude_bin': 0.5,
               'time_bin': 5.,
               'increment_lock': True}
\end{python}

The algorithm has three configurable options. The \verb=time_bin= parameter describes the size of the time window in years, the \verb=magnitude_bin= parameter describes the size of the magnitude bin, sensitivity is as described previously. The final option (\verb=increment_lock=) is an option that is used to ensure consistency in the results to avoid the completeness magnitude increasing for the latest intervals in the catalogue simply due to the variability associated with the short duration. If \verb=increment_lock= is set to \verb=True=, the program will ensure that the completeness magnitude for shorter, more recent windows is less than or equal to that of older, longer windows. This is often a condition for some recurrence analysis tools, so it may be advisable to set this option to true in certain workflows. Otherwise it should be set to \verb=False to show the apparent variability=. Some degree of judgement is necessary here. In particular it is expected that the user may be aware of circumstances particular to their catalogue for which a recent increase in completeness magnitude is expected (for example, a certain recording network no longer operational).  


The process of running the algorithm is shown below:

\begin{python}[frame=single]
>> from openquake.hmtk.seismicity.completeness.comp_stepp_1971 import\
    Stepp1971

>> completeness_algorithm = Stepp1971()

>> completeness_table = completeness_algorithm.completeness(
    catalogue,
    comp_config)

>> completeness_table 
array([[ 1990.  ,     4.25],
       [ 1962.  ,     4.75],
       [ 1959.  ,     5.25],
       [ 1906.  ,     5.75],
       [ 1906.  ,     6.25],
       [ 1904.  ,     6.75],
       [ 1904.  ,     7.25]])
\end{python}

As shown in the resulting \verb=completeness_table=, the completeness algorithm will output the time variation in completeness (in this example with the \verb=increment_lock= set) in the form of a two-column table with column 1 indicating the completeness year for the magnitude bin centred on the magnitude value found in column 2.

At present, it may be the case that the user wishes to enter a time-varying completeness results for use in subsequent functions, based on alternative methods or on judgement. This can be entered in the \verb=completeness_table= setting, as in the example shown here (take note of the requirements for the square brackets):

\begin{python}[frame=single]
completeness_table: [[1990., 4.0],
                     [1960., 5.0],
                     [1930., 6.0],
                     [1900., 6.5]]
\end{python}

If a \verb=completeness_table= is input then this will override the selection of the completeness algorithm, and the calculation will take the values in \verb=completeness_table= directly. 

%::::::::::::::::::::::::::::::::::::::::::::::::::::::::::::::::::::::::::::::::::::::::::::::::::::::::::::::::::::::::::::::::::::::::::::::::::::::::::::::

\section{Recurrence Models}

The current sets of tools are intended to determine the parameters of the \textcite{GutenbergRichter1944} recurrence model, namely the a- and b-value. It is expected that in the most common use case the catalogue that is input to these algorithms will be declustered, with a time-varying completeness defined according to a \verb=completeness_table= of the kind shown previously. If no \verb=completeness_table= is input the algorithm will assume the input catalogue is complete above the minimum magnitude for its full duration.

\subsection{\textcite{Aki1965}}

The classical maximum likelihood estimator for a simple unbounded \textcite{GutenbergRichter1944} model is that of \textcite{Aki1965}, adapted for binned magnitude data by \textcite{Bender1983}. It assumes a fixed completeness magnitude ($M_C$) for the catalogue, and a simple power law recurrence model. It does not explicitly take into account magnitude uncertainty.

\begin{equation}
   b = \frac{ \log_{10} \left( e \right)}{ \bar{m} - m_0 + \left( {\frac{\Delta M}{2}} \right)}
\end{equation}

\noindent where $\bar{m}$ is the mean magnitude, $m_0$ the minimum magnitude and $\Delta M$ the discretisation interval of magnitude within a given sample.

\subsection{Maximum Likelihood}

This method adjusts the \textcite{Aki1965} and \textcite{Bender1983} method to incorporate for time variation in completeness. The catalogue is divided into
into S sub-catalogues, where each sub-catalogue corresponds to a period 
with a corresponding $M_C$.  An average a- and b-value (with uncertainty) is returned by taking 
the mean of the a- and b-value of each sub-catalogue, weighted by 
the number of events in each sub-catalogue.

\begin{equation}
   \hat{b} = \frac{1}{S} \sum_{i = 1}^{S} w_i b_i
\end{equation}

\begin{python}[frame=single]
>> mle_config = {'magnitude_interval': 0.1,
                 'Average Type': 'Weighted',
                 'reference_magnitude': None}

>> from openquake.hmtk.seismicity.occurrence.b_maximum_likelihood import\
    BMaxLikelihood

>> recurrence = BMaxLikelihood()

>> bval, sigmab, aval, sigmaa = recurrence.calculate(
    catalogue,
    mle_config, 
    completeness=completeness_table)
\end{python}

Where \verb=magnitude_window= indicates the size of the magnitude bin, \verb=recurrence_algorithm= and \verb=reference_magnitude= the magnitude for which the output calculates that rate greater than or equal to (set to \verb=0= for $10^{a}$). 

\subsection{\textcite{KijkoSmit2012}}

A recent adaption of the \textcite{Aki1965} estimator of b-value for  a catalogue containing different completeness periods has been proposed by \textcite{KijkoSmit2012}. Dividing the earthquake catalogue into $s$ subcatalogues of $n_i$ events with corresponding completeness magnitudes $m_{c_i}$ for $i = 1, 2, ..., s$, the likelihood function of $\beta$ where $\beta = b \ln		 \left( {10.0} \right)$ is given as:

\begin{equation}
    \mathbf{L} = \prod_{i = 1}^{s} \prod_{j = 1}^{n_i} \beta \exp(\left[ {-\beta \left( {m_j^i - m_{min}^i } \right) } \right])
\end{equation}

\noindent which gives a maximum likelihood estimator of $\beta$:

\begin{equation}
    \beta = \left( {\frac{r_1}{\beta_1} + \frac{r_2}{\beta_2} + \dots + \frac{r_s}{\beta_s}} \right)^{-1}
\end{equation}

\noindent where $r_i = n_i / n$ and $n = \sum_{i = 1}^{s} n_i$ above the level of completeness $m_i$.

\begin{python}[frame=single]]

>> kijko_smit_config = {'magnitude_interval': 0.1,
                        'reference_magnitude': None\}
\end{python}

\subsection{\textcite{Weichert1980}}

Recognising the typical conditions of an earthquake catalogue, \textcite{Weichert1980} developed a maximum likelihood estimator of $b$ for grouped magnitudes and unequal periods of observation. The likelihood formulation for this approach is:

\begin{equation}
   \mathbf{L} \left( {\beta | n_i, m_i, t_i} \right) = \frac{ N!}{\prod_i n_i!} \prod_i p_{i}^{n_i}
\end{equation}

where $\mathbf{L}$ is the likelihood estimator of $\beta$, $n$ the number of earthquakes in magnitude bin m with observation period t. The parameter $p$ is defined as:

\begin{equation}
   p_i = \frac{t_i \exp \left( {-\beta m_i} \right) }{\sum_j t_j \exp \left( {-\beta m_j} \right)}
\end{equation}

The extremum of $\ln \left( {\mathbf{L}}\right)$ is found at:

\begin{equation} 
   \frac{\sum_i t_i m_i \exp \left( {-\beta m_i} \right)}{\sum_j t_j \exp \left( {-\beta m_j} \right)}
\end{equation}

The computational implementation of this method is given as an appendix to \textcite{Weichert1980}. This formulation of the maximum likelihood estimator for b-value, and consequently seismicity rate, is in widespread use, with applications in many national seismic hazard analysis \parencite[e.g.][]{usgsNSHM1996,usgsNSHM2002}. The algorithm has been demonstrated to be efficient and unbiased for most applications. It is recognised by \textcite{Felzer2008} that an implicit assumption is made regarding the stationarity of the seismicity for all the time periods. 

To implement the \textcite{Weichert1980} recurrence estimator, the configuration properties are defined as:

\begin{python}[frame=single]

>> weichert_config = {`magnitude_interval': 0.1,
                      `reference_magnitude': None,
                      # The remaining parameters are optional
                      `bvalue': 1.0,
                      `itstab': 1E-5,
                      `maxiter': 1000}
\end{python}

As the \textcite{Weichert1980} algorithm is reaches the MLE estimation by iteration then three additional optional parameters can control the iteration process: \verb=bvalue= is the initial guess for the b-value, \verb=itstab= the difference in b-value in order to reach convergence, and \verb=maxiter= the maximum number of iterations. \footnote{The iterative nature of the \textcite{Weichert1980} algorithm can result in very slow convergence and unstable behaviour when the magnitudes infer b-values that are very small, or even negative. This can occur when very few events are in the resulting catalogue, or when the magnitudes converge within a narrow range.}



%::::::::::::::::::::::::::::::::::::::::::::::::::::::::::::::::::::::::::::::::::::::::::::::::::::::::::::::::::::::::::::::::::::::::::::::::::::::::::::::
\section{Maximum Magnitude}

The estimation of the maximum magnitude for use in seismic hazard analysis is a complex, and often controversial, process that should be guided by information from geology and the seismotectonics of a seismic source. Estimation of maximim magnitude from the observed (instrumental and historical) seismicity can be undertaken using methods assuming a truncated \cite{GutenbergRichter1944} model, or via non-parametric methods that are independent any assumed functional form. 

\subsection{\textcite{Kijko2004}}

Three different estimators of maximum magnitude are given by \textcite{Kijko2004}, each depending on a different set of assumptions:
\begin{enumerate}
\item ''Fixed b-value'': Assumes a single b-value with no uncertainty 
\item ''Uncertain b-value'': Assumes and uncertain b-value defined by an expected b and the standard deviation
\item ''Non-Parametric Gaussian'': Assumes no functional form (can be applied to seismicity observed to follow a more characteristic distribution)
\end{enumerate}

Each of these estimators assumes the general form:

\begin{equation}
m_{max} = m_{max}^{obs} + \Delta
\end{equation}

where $\Delta$ is an increment that is dependent on the estimator used.

The uncertainty on $m_{max}$ is also defined according to:

\begin{equation}
    \sigma_{m_{max}} = \sqrt{\sigma_{m_{max}^{obs}}^2 + \Delta^{2}}
\end{equation}

In the three estimators some lower bound magnitude constraint must be defined. For those estimators that assume an exponential recurrence model the lower bound magnitude must be specified by the users. For the non-Parametric Gaussian method and explicit lower bound magnitude does not have to be specified; however, the estimation is conditioned upon the largest N magnitudes, where N must be specified by the user.

If the user wishes to input a maximum magnitude that is larger than that observed in the catalogue (e.g. a known historical magnitude), this can be specified in the config file using \verb=input_mmax= with the corresponding uncertainty defined by \\ \verb=input_mmax_uncertainty=. If these are not defined (i.e. set to \verb=None=) then the maximum magnitude will be taken from the catalogue.

All three estimators require an iterative solution, therefore additional parameters can be specified in the configuration file that control the iteration process: \verb=tolerance= difference in $M_Max$ estimate for the algorithm to be considered converged, and \\ \verb=maximum_iterations= the maximum number of iterations for stability. 


\subsubsection{''Fixed b-value''}

For a catalogue of $n$ earthquakes, whose magnitudes are distributed by a \textcite{GutenbergRichter1944} distribiution with a fixed "b" value, the increment of maximum magnitude is determined via:

\begin{equation}
\Delta = \int\limits_{m_{min}} ^{m_{max}} \left[ {\frac{1 - \exp \left[ {-\beta \left( {m - m_{min}} \right)} \right]}{1 - \exp \left[ {-\beta \left( {m_{max}^{obs} - m_{min}} \right) } \right]}} \right] ^n dm
\end{equation}


The execution of the \textcite{Kijko2004} ''fixed-b'' algorithm is as follows:

\begin{python}[frame=single]
>> mmax_config = {`input_mmax': 7.6,
                  `input_mmax_uncertainty': 0.22,
                  `b-value': 1.0,
                  `input_mmin': 5.0,
                  `tolerance': 1.0E-5,  \# Default
                  `maximum_iterations': 1000\} \# Defaults
                       
>> from openquake.hmtk.seismicity.max_magnitude.kijko_sellevol_fixed_b\
    import KijkoSellevolFixedb

>> mmax_estimator = KijkoSellevolFixedb()

>> mmax, mmax_uncertainty = mmax_estimator.get_mmax(catalogue,
                                                    mmax_config)   
\end{python}

\subsubsection{''Uncertain b-value''}

For a catalogue of $n$ earthquakes, whose magnitudes are distributed by a \textcite{GutenbergRichter1944} distribiution with an uncertain "b" value, characterised by and expected term ($b$) and a corresponding undertainty ($\sigma_b$), the increment of maximum magnitude is determined via:


\begin{equation}
\Delta = \left( {C_{\beta}} \right)^n \int\limits_{m_min}^{m_max} \left[ {1 - \left( {\frac{p}{p + m - m_{min}}} \right) ^q} \right]^n dm
\end{equation}

where $\beta = b \ln \left( {10.0} \right)$, $p = \beta / \left( {\sigma_{\beta}} \right) ^ 2$, $q = \left( {\beta / \sigma_{\beta}} \right) ^ 2$ and $C_{\beta}$ is a normalising coefficient determined via:

\begin{equation}
C_{\beta} = \frac{1}{1 - \left[ {p / \left( {p + m_{max} - m_{min}} \right) } \right]^q}
\end{equation}

In both the fixed and uncertain ''b'' case a minimum magnitude will need to be input into the calculation. If this value is lower than the minimum magnitude observed in the catalogue the iterator may not stabilise to a satisfactory value, so it is recommended to use a minimum magnitude that is greater than the minimum found in the observed catalogue.

The execution of the ''uncertain b-value'' estimator is undertaken in a very similar to that of the fixed b-value, the only additional parameter being the \verb=sigma-b= term:

\begin{python}[frame=single]
>> mmax_config = {'input_mmax': 7.6,
                  'input_mmax_uncertainty': 0.22,
                  'b-value': 1.0,
                  'sigma-b': 0.15
                  'input_mmin': 5.0,
                  'tolerance': 1.0E-5, 
                  'maximum_iterations': 1000}
                       
>> from openquake.hmtk.seismicity.max-magnitude.kijko_sellevol_bayes\
    import KijkoSellevolBayes

>> mmax_estimator = KijkoSellevolBayes()

>> mmax, mmax_uncertainty = mmax_estimator.get_mmax(catalogue,
                                                    mmax_config)
\end{python}

\subsubsection{Non-Parametric Gaussian}

The non-parametric Gaussian estimator for maximum magnitude $m_{max}$ is defined as:

\begin{equation}
\Delta = \int\limits_{m_{min}}^{m_{max}} \left[ {\frac{\sum_{i = 1}^{n} \left[ {\Phi \left( {\frac{m - m_i}{h}} \right) - \Phi \left( {\frac{m_{min} - m_i}{h}} \right)} \right]}{\sum_{i = 1}^{n} \left[ {\Phi \left( {\frac{m_{max} - m_i}{h}} \right) - \Phi \left( {\frac{m_{min} - m_i}{h}} \right)} \right]}} \right]^n  dm
\end{equation}
where $m_{min}$ and $m_{max}$ are the minimum and maximum magnitudes from a set of $n$ events, $\Phi$ is the standard normal cumulative distribution function. $h$ a kernel smoothing factor:
\begin{equation}
h = 0.9 \times min\left( {\sigma, IQR / 1.34} \right) \times n^{-1 / 5}
\end{equation}
with $\sigma$ the standard deviation of a set of n earthquakes with magnitude $m_{i}$ where $i = 1, 2, ... n$, and $IQR$ the inter-quartile range. 

Therefore the uncertainty on $m_{max}$ is conditioned primarily on the uncertainty of the largest observed magnitude. As in many catalogues the largest observed magnitude may be an earlier historical event, which will be associated with a large uncertainty, this estimator tends towards large uncertainties on $m_{max}$.

Due to the need to define some additional parameters the configuration file is slightly different. No b-value or minimum magnitude needs to be specified; however, the algorithm will consider only the largest \verb=number_earthquakes= magnitudes (or all magnitudes if the number of observations is smaller). The algorithm also numerically approximates the integral of the Gaussian pdf, so \verb=number_samples= is the number of samples of the distribution. The rest of the execution remains the same as for the exponential recurrence estimators of $M_{max}$:

\begin{python}[frame=single]
>> mmax_config = {'input_mmax': 7.6,
                  'input_mmax_uncertainty': 0.22,
                  'number_samples': 51, # Default
                  'number_earthquakes': 100 # Default 
                  'tolerance': 1.0E-5, 
                  'maximum_iterations': 1000}
                       
>> from openquake.hmtk.seismicity.max-magnitude.kijko_nonparametric_gaussian\
     import KijkoNonParametricGaussian

>> mmax_estimator = KijkoNonParametricGaussian()

>> mmax, mmax_uncertainty = mmax_estimator.get_mmax(catalogue,
                                                    mmax_config)
                
\end{python}


\subsection{Cumulative Moment \parencite{MakropoulosBurton1983}}

The cumulative moment release method is an adaptation of the cumulative strain energy release method for estimating $m_{max}$ originally proposed by \textcite{MakropoulosBurton1983}. Another method based on a pseudo-graphical formulation, an estimator of maximum magnitude can be derived from a plot of cumulative seismic moment release with time. The average slope of this plot indicates the mean moment release for the input catalogue in question. Two further straight lines are defined with gradients equal to that of the slope of mean cumulative moment release, both enveloping the cumulative plot. The vertical distance between these two lines indicates the total amount of moment that may be released in the region, if no earthquakes were to occur in the corresponding time (i.e. the distance between the upper and lower bounding lines on the time axis). This concept is illustrated in Figure \ref{fig:Cumulative_Moment}. 

\begin{figure}[htb]
	\centering
		\includegraphics[height=6cm, keepaspectratio=true]{./figures/Cumulative_Moment.eps}
	\caption{Illustratation of Cumulative Moment Release Concept}
	\label{fig:Cumulative_Moment}
\end{figure}

The cumulative moment estimator of $m_{max}$, whilst simple in concept, has several key advantages. As a non-parametric method it is independent of any assumed probability distribution and cannot estimate $m_{max}$ lower than the observed $m_{max}$. It is also principally controlled by the largest events in the catalogue, this making it relative insensitive to uncertainties in completeness or lower bound threshold. In practice, this estimator, and to some extent that of \textcite{Kijko2004} are dependent on having a sufficiently long record of events relative to the strain cycle for the region in question, such that the estimate of average moment release is stable. This will obviously depend on the length of the catalogue, and for some regions, particularly those in low strain intraplate environments, it is often the case that $m_{max}$ will be close to the observed $m_{max}$. Therefore it may be the case that it is most appropriate to use these techniques on a larger scale, either considering multiple sources or an appropriate tectonic proxy.

For the cumulative moment estimator it is possible to take into account the uncertainty on $m_{max}$ by applying bootstrap sampling to the observed magnitudes and their respective uncertainties. This has the advantage that $\sigma_{m_{max}}$ is not controlled by the uncertainty on the observed $m_{max}$, as it is for the \textcite{Kijko2004} algorithm. Instead it takes into account the uncertainty on all the magnitudes in the catalogue. The cost of this, however, is that this method is more computationally intensive, and therefore slower, than \textcite{Kijko2004}, depending on the number of bootstrap samples the user chooses.

The algorithm is slightly simpler to run than the \textcite{Kijko2004} methods; however, due to the bootstrapping process it is slightly slower. It is run as per the following example:

\begin{python}[frame=single]

>> mmax_config = {'number_bootstraps': 1000}
                       
>> from openquake.hmtk.seismicity.max_magnitude.cumulative_moment_release\
     import CumulativeMoment

>> mmax_estimator = CumulativeMoment()

>> mmax, mmax_uncertainty = mmax_estimator.get_mmax(catalogue,
                                                    mmax_config)
                
\end{python}

For the cumulative moment algorithm the only user configurable parameter is the \\ \verb=number_bootstraps=, which is the number of samples used during the bootstrapping process. 

\section{Smoothed Seismicity}

The use of smoothed seismicity in seismic hazard analysis has generally become a common way of characterising distributed seismicity, for which the seismogenic source are defined exclusively from the uncertain locations of observed seismicity. There are many different methods for smoothing the catalogue, adopting different smoothing kernels or making different correction factors to compensate for spatial and/or temporal completeness. 

\subsection{\textcite{frankel1995}}

A smoothed seismicity method that has one of the clearest precedents for use in seismic hazard analysis is that of \textcite{frankel1995}, originally derived to characterise the seismicity of the Central and Eastern United States as part of the 1996 National Seismic Hazard Maps of the United States. The method applies a simple isotropic Gaussian smoothing kernel to derive the expected rate of events at each cell $\tilde{n}_i$ from the observed rate $n_j$ of seismicity in a grid of $j$ cells. This kernel takes the form:

\begin{equation}
\tilde{n_i} = \frac{\sum_j n_j e^{d_{ij}^2 / c^2}}{\sum_j e^{d_{ij}^2 / c^2}} 
\end{equation}

In the implementation of the algorithm, two steps are taken that we prefer to make configurable options here. The first step is that the time-varying completeness is accounted for using a correction factor ($t_f$) based on the \textcite{Weichert1980} method:

 \begin{equation}
 t_f = \frac{\sum_i e^{-\beta m_{c_i}}}{\sum_i T_i e^{-\beta m_{c_i}}} 
 \end{equation}
 
where $m_{c_i}$ the completeness magnitude corresponding to the mid-point of each completeness interval, and $T_i$ the duration of the completeness interval. The completeness magnitude bins must be evenly-spaced; hence, within the application of the progress a function is executed to render the input completeness table to one in which the magnitudes are evenly spaced with a width of 0.1 magnitude units. 

\subsection{Implementing the Smoothed Seismicity Analysis}

The smoothed seismicity separates out the core implementation (i.e. the gridding, counting and execution of the code) and the choice of kernel. An example of the execution process is as follows:

The first stage is to upload the catalogue into an instance of the catalogue class

\begin{python}[frame=single]
>> input_file = 'path/to/input_file.csv'

>> from openquake.hmtk.parsers.catalogue.csv_catalogue_parser import\
    CsvCatalogueParser

>> parser = CsvCatalogueParser(input_file)

>> catalogue = parser.read_file()
\end{python}

Next setup the smoothing algorithm using and the corresponding kernel:

\begin{python}[frame=single]

# Imports the smoothed seismicity algorithm
>> from openquake.hmtk.seismicity.smoothing.smoothed_seismicity import\
    SmoothedSeismicity

# Imports the Kernel function
>> from openquake.hmtk.seismicity.smoothing.kernels.isotropic_gaussian\
    import IsotropicGaussian

# Grid limits should be set up as 
# [min_long, max_long, spc_long, 
#  min_lat max_lat, spc_lat,
#  min_depth, max_depth, spc_depth]
>> grid_limits = [0., 10., 0.1, 0., 10., 0.1, 0., 60., 30.]
# Assuming a b-value of 1.0
>> smooth_seis = SmoothedSeismicity(grid_limits,
                                    use_3d=True,
                                    bvalue=1.0)
\end{python}

The smoothed seismicity function needs to be set up with three variables: i) the extent (and spacing) of the grid, ii) the choice to use 3D smoothing (i.e. distances are taken as hypocentral rather than epicentral) and iii) the input b-value. The extent of the grid can also be defined from the catalogue. If preferred the user need only specify the spacing of the longitude-latitude grid (as a single floating point value), then the grid will be defined by taking the bounding box of the earthquake catalogue and extended by the total smoothing length (i.e. the bandwidth (in km) multiplied by the maximum number of bandwidths). 

To run the smoothed seismicity analysis, the configurable parameters are: \verb=BandWidth= the bandwidth of the Gaussian kernel (in km), \verb=Length_Limit= the number of bandwidths considered as a maximum smoothing length, and \verb=increment= chooses whether to output the incremental a-value (for consistency with the original \textcite{frankel1995} methodology) or the cumulative a-value (corresponding to the a-value of the Gutenberg-Richter model).


The algorithm requires two essential inputs (the earthquake catalogue and the config file), and three optional inputs:

\begin{itemize}
\item \verb=completeness_table= A table of completeness magnitudes and their corresponding completeness years (as output from the completeness algorithms)

\item \verb=smoothing_kernel= An instance of the required smoothing
kernel class (currently only Isotropic Gaussian is supported - and will be used if not specified)

\item \verb=end_year= The final year of the catalogue. This will be taken as the last year found in the catalogue, if not specified by the user
\end{itemize}

The analysis is then run via:

\begin{python}[frame=single]
# Set up config (e.g. 50 km band width, up to 3 bandwidths)
>> config = {`Length_Limit': 3.,
             `BandWidth': 50.,
             'increment': True}
# Run the analysis!
>> output_data = smooth_seis.run_analysis(
    catalogue,
    config,
    completeness_table, 
    smoothing_kernel=IsotropicGaussian(), 
    end_year=None)

# To write the resulting data to a csv file
>> smooth_seis.write_to_csv(`path/to/output_file.csv')
\end{python}

The resulting output will be a csv file with the following columns:
\begin{Verbatim}[frame=single, commandchars=\\\{\}, fontsize=\scriptsize]
Longitude, Latitude, Depth, Observed Count, Smoothed Rate, b-value
\end{Verbatim}

\noindent where \verb=Observed Count= is the observed number of earthquakes in each cell, and \\ 
\verb=Smoothed Rate= is the smoothed seismicity rate.

%\cleardoublepage
%
%
%% \cleardoublepage
%% %
%\chapter{Hazard Tools}
%\begin{myfancybox}
%The objectives of this chapter are:
%\begin{itemize}
%    \item Introduction to the HMTK source model classes
%    \item Understand how to link source models to seismicity
%    \item Run simple PSHA calculations in OpenQuake, using the HMTK
%\end{itemize}
%\end{myfancybox}
%  \input{hazard_tools.tex}
%\cleardoublepage
% 
%
%-----------------------------------------------------------------------------
%\chapter{Geological Tools}
%\begin{myfancybox}
%The objectives of this chapter are:
%\begin{itemize}
%    \item Describe features of the geological tools Hazard Modeller's Toolkit 
%    \item Run simple calculations using the geological tools 
%\end{itemize}
%\end{myfancybox}
%  \input{geology_tools.tex}
%\cleardoublepage
%
%
%% \cleardoublepage
%% % -----------------------------------------------------------------------------
%\chapter{Geodetic Tools}
%\begin{myfancybox}
%The objectives of this chapter are:
%\begin{itemize}
%    \item Describe features of the geodetic tools Hazard Modeller's Toolkit 
%    \item Run simple calculations using the geodetic tools 
%\end{itemize}
%\end{myfancybox}
%  \input{geodetic_tools.tex}
%\cleardoublepage


% -----------------------------------------------------------------------------

% -----------------------------------------------------------------------------
% -----------------------------------------------------------------------------
%%\cleardoublepage
%\bibliographystyle{apalike}
%\bibliography{./bibliography/hazard}
%\cleardoublepage
%\printglossaries
%\printindex
%% -----------------------------------------------------------------------------
% --------------------------------------------------------------------------
